\chapter{History and Definition of Mysticism}

\section{Early Rites and Etymons}

\subsection{Primitive History and Etymology}

Philippe De Felice, in \emph{Poisons Sacrés. Ivresses Divines},
wrote, concerning the immemorial connection between certain
forms of intoxication and religion, that \enquote{The practices
studied in this volume can be observed in every region of
the Earth, among primitives no less than among those who
have reached a high pitch of civilization. We are, therefore,
dealing not with exceptional facts\dots, but with a general
and, in the widest sense of the word, a human phenomenon, the
kind of phenomenon which cannot be disregarded by anyone who
is trying to discover what religion is, and what are the
deep needs which it must satisfy.} Also, the \emph{Encyclopedia
of Philosophy and Religion} includes a discussion of the
primitive religious forms of mana, shamanism, fetishism, and
the traces of medicine men, in which intoxication generally
plays a large part, in its discussion of mysticism.\footnote{\enquote{Most forms of shamanism come within the sphere of mysticism.}\supercite{hastings:philosophy-religion}}
Although many mystics, like Blake, seem to have been \enquote{drunk with
intellectual vision,} rather than with drugs, this association
of drunkenness with \enquote{divine experience} seems to be universal,
except among those extreme dualists, or both the East and the
West, who reject consciousness entirely, and in such cases as
these it is only by their claims, and the claims of their
scholars, that they can be considered mystics:\footnote{Besides the well known shamanistic rites of Siberia, \enquote{divine mushrooms} are known to be used in Borneo, New Guinea, and Mexico, and, at least according to tradition, in China, Japan, and India.\supercite{deropp:drugs-mind, wasson:magic-mushroom} \enquote{\dots there is good reason to assume the existence of a sacred-mushroom cult in Ancient Egypt.}\supercite{puharich:sacred-mushroom}}
the implication is, of course, that \enquote{mysticism} is in some way innately
connected with \enquote{intoxication.} R. Gordon Wasson has written
that there is evidence which links the Greek word \emph{μύσται} with
the root \emph{myoo}, fungus: that fungi, especially mushrooms, are
used as intoxicants is widely known to anthropologists.
It is known, regardless of what explanation, if any, is
given, that natural and related languages contain startling
similarities to European languages; for example, the root
\emph{teo} as used, for instance, in the name of the sacred mushroom,
\emph{teonanacatl}, \enquote{God's flesh,} is obviously similar to the Greek
\emph{Theos} and the Latin \emph{Deus}. Although it cannot be claimed to
be more than an extreme coincidence, a member of the tribe
which uses the sacred mushroom is called a \emph{Mixtec}, which, when
pronounced in Nahuatl, is \emph{identical} in sound to the English
word \enquote{mystic.}

\subsection{Greek Parallels to the Primitive Rites}

Support for Wasson's theory is found in the \enquote{Song to
Demeter} in the \emph{Homeric Hymns},\supercite{hesiod:homeric-hymns}
where Rhea says to Demeter (\emph{μήτηρ} means \enquote{mother,} so Demeter probably represents
\enquote{Mother Earth}): \enquote{\enquote*{But come, my child, obey, and be not
too angry unrelentingly with the dark-clouded son of Conos;
but rather increase forthwith for men the fruit that gives
them life.} So spoke Rhea, and rich-crowned Demeter did
not refuse but straightway made fruit to spring up from the
rich lands.} Although this could, of course, refer to mere
vegetable nourishment, the lines immediately following these
seem to make it clear that it was some sort of vegetable
intoxicant that was gathered by the initiates on their over-night
trip to the countryside, and that was carried back in
a box, from which it would be taken by each initiate:\footnote{\enquote{\dots I have taken (the things) from the sacred chest, having tasted thereof\dots}\supercite{britannica:encyclopedia}}
\enquote{Then she went, and to the kings who deal justice, Triptolemus
and Diocles, the horse-driver, and to doughty Eumolpus and
Celeus, leader of the people, she showed the conduct of the
rites and taught them all her mysteries, to Triptolemus and
Polyxeinus and Diocles also,---awful mysteries which no
one may in any way transgress or pry into or utter, for
deep awe of the gods checks the voice.}\supercite{hesiod:homeric-hymns}
Although \emph{The Encyclopedia of Philosophy and Religion} indicates that, in
Cicero's view, Athens produced \enquote{nothing better than the
mysteries of Eleusis, not only in regard to the ordering and
civilizing of life, but in regard to the furnishing of a good
hope in death,} and that Sophocles believed that \enquote{happiness
in the next world} was confined to \enquote{\dots those who had been
initiated in the mysteries of Eleusis, that is, probably
so far as their fellow-citizens were concerned.}\supercite{hastings:philosophy-religion}
The \enquote{Song to Demeter} does not explicitly impute a doctrine or \enquote{after-
life} to the mysteries: \enquote{Happy is he among men on Earth who
has seen these mysteries; but he who is uninitiate and who
has no part in them, never has lot of liken good things once
he is dead, down in the darkness and gloom.}\supercite{hesiod:homeric-hymns}
This statement, it seems, could be interpreted to represent either a denial
of after-life for those who are not initiated, with no
reference to the future of the initiate, or a statement that
the initiate will have a superior after-life.

\section{Oriental Rites}

\subsection{Hinduism}

The doctrine of the after-life was highly developed in
Greece, regardless of whether a particular \enquote{mystery} contained
the idea, and Pythagoras, or rather Pythagoreans, with
their ideas of the \enquote{wheel of birth} and the transmigration of souls\supercite{britannica:encyclopedia}
(both of which involved the idea of karma, however
it was expressed) seem to have been representative of this
doctrine, as classical Hinduism, with the same particular
ideas, was representative of that doctrine in India. Although
little seems to be known about the actual practices
of \enquote{purification} of the Pythagorean Brotherhood, they can
be considered mystical, by the same criteria which allow
the \emph{Encyclopedia of Philosophy and Religion} to assert that
\enquote{In Hinduism, indeed, in nearly all of its manifestations,
in its most philosophical flights as well as when it
approaches pure shamanism and magic, there are to be found
indications of the mystical temper of mind.}\supercite{hastings:philosophy-religion}\footnote{See also pp. \pageref{self:01} of this thesis a similar statement concerning the classification of Buddhism.}

Although Hinduism is widely considered to be an ascetic religion, 
the \emph{Bhagavad-Gītā} contains what seems to be an
early doctrinal justification for the \enquote{modernized} forms of
the religion, which were modified by the influence of the
primitive Naga and Yaksa worship and the competition of
liberalized Buddhism. The \emph{Gītā} says that some \emph{Yogis} offer
their soul to the \enquote{sacrificial fire} of Brahman, and others
renounce possessions, activity, and sense-perception; however,
immortality and the reaching of \enquote{eternal Brahman,}
besides happiness \enquote{in this world,} can also be obtained by
those who \enquote{allow their minds and senses to wander unchecked,
and try to see Brahman within all exterior sense-objects.
For these, sound and the other sense-objects are the offering,
and sense-enjoyment the sacrificial fire.}\supercite{vyasa:bhagavad-gita}

\phantomsection
\label{self:17}

\subsection{Mature Mahayana Buddhism}

Since the classical Hindu tradition of renunciation is both very well
known and practically identical with the main Christian
mystical tradition, and since the Tantric Hindu movement
closely parallels the \enquote{liberal} Buddhist practices, differing
mainly in its attempt to \enquote{\dots assimilate and adjust itself to
the Orthodox tradition (rather) than to exclude and refute it.}\supercite{heinrich:philosophies-india}
It will be assumed that the discussions of the mature
Mahayana Buddhism and the Catholic Christian mystical tradition
will illustrate the two main types of mysticism sufficiently.

\phantomsection
\label{self:01}

The fact that Buddhism has a greater reputation for being
monistic rather than Hinduism is revealed by the statement in the
\emph{Encyclopedia of the Social Sciences} that Buddhism has spread
into areas which had previously been Christian, Taoist,
Zoroastrian, and even Brahmnist because of its practical
nature, its goals being primarily humanitarian, and psychology
being the means used to achieve them. According to the definition
given by the above, Buddhism can be considered to be
\enquote{mystical} if it makes use of such practices as \enquote{ecstasy,}
\enquote{trance,} and \enquote{consciousness of the absolute,} so it can be
seen that both of these statements about Buddhism can be
simultaneously correct of the mentioned \enquote{practices} are
understood in a monistic sense, and if \enquote{humanitarian} is
not defined too narrowly. The monistic Buddhist might say
that \enquote{through certain trances or exercises man can, in
accordance with the Four Noble Truths,\footnote{(1) The truth of pain, (2) The truth of the cause of pain, (3) The truth of cessation of pain, and (4) The truth of the way that leads to the cessation of pain.\supercite{noss:mans-religions}}
overcome ignorance, and thereby become aware of the highest truth and experience
the highest delight, Maha-Sukha.} The person who desires to
\enquote{overcome his ignorance,} that is, to be initiated, need
only, as in the parallel Hindu and Greek religions, be
\enquote{\dots intelligent,\dots one who abstains from injuring any being,
ever doing good to all, pure, \dots and a nondualist\dots,}\supercite{heinrich:philosophies-india}
That is, women and slaves were not excluded, and could even
become \emph{Gurus}, or teachers of the religion.

The \enquote{metaphysical} explanation of this aspect of Buddhism
is given by Zimmer as follows:

\quotebox{%
    \dots pure compassion is of the essence of the Bodhisattva
    and is  identical with his right perception of the void;
    or, as one might say, it is the primary reflex of the void.\supercite{heinrich:philosophies-india}
    \dots \enquote{The Bodhisattva attains omniscience.}\supercite{heinrich:philosophies-india}\footnotemark
    \ Within the hearts of all creatures compassion is present
    as the sign of their potential Bodhisattvahood; for all things
    are Sunyata, the void---and the pure reflex of this void (which
    is their essential being) is compassion. Compassion (\emph{Karuna})
    indeed is the forge that holds things in manifestation---just
    as it withholds the Bodhisattva from Nirvana.
}

\footnotetext{%
    It is important to note that \emph{all} of those who hold an entirely
    monistic mystical doctrine maintain that \enquote{omniscience} is characteristic
    of the person who has attained the condition of \enquote{inspired perception} or
    \enquote{four-fold vision}: examples are the Mixtec Indians, the Hindus, the Buddhists
    being discussed here, and Blake, who said \enquote{less than all cannot satisfy man.}\supercite{keynes:william-blake}
}

This \enquote{Nirvana} from which the Bodhisattva is withheld by his
compassion refers to the Nirvana of the dualistic mystics,
that is, the ineffable and utterly transcendent reality; the
condition being described above may also be called \enquote{Nirvana,}
but it might be more accurate to limit its name to that used
by the parallel Hindu sects, viz., \enquote{\emph{Mahanirvana}.}\supercite{heinrich:philosophies-india}
Zimmer continued his explanation of monistic Buddhism with an indication
of its contrast to Christianity, Vedantism, and Hinayana
Buddhism, as follows:

\quotebox[2.25em]{%
    \dots this world-supporting condescension of the Bodhisattva\dots in
    spirit and practice\dots takes us one step further (than
    the Christian mystery of the incarnation), since it calls for
    \emph{an unqualified affirmation of \enquote{ignorance}} (Avidya) \emph{as in
    essence identical with \enquote{enlightenment}} (body)---which renders
    archaic the ancient Sankhya-Vedanta-Hinayana modes of Monkism
    rejection or acceptance\dots \enquote{ignorance} (Avidya) is still\dots the
    cause of suffering\dots the benighting affliction of those
    who live in desire and fear, in hope, despair, disgust, and
    sorrow. But the one whose mind is cleansed, whose \enquote{soul,}
    whose selfhood, has become annihilate in the void, he is identical\dots
    mingled with the compassion of the Bodhisattva is a quality,
    therefore, of \enquote{great delight} (Maha-Sukha)\dots hence the
    Bodhisattva wanders everywhere, boundless, fearless, like a lion,\footnotemark
    roaring the lion-roar of Bodhisattva, these three worlds have been
    created, as it were, for---by---and of---the enjoyment of this
    immortal: they are his \emph{Lila}, his \enquote{play.}
}

\footnotetext{See page \pageref{self:02} of this thesis.}

\phantomsection
\label{self:27}

\quotebox[2.25em]{%
    Since the candidate for such knowledge must behave like
    one who has already attained, a programmatic, sacramental
    breaking of the bounds that normally stand as the limits of
    virtue was carefully undertaken in certain schools of the
    Mahayana. In spite of all the scandal that has been spread
    concerning this phase of Buddhist worship, the majority of
    the sacramental breaches (in a society hedged on every side
    by the most meticulous taboos) were not such as would give
    the slightest pause to the usual modern Christian gentleman or lady.
    They consisted in partaking of such forbidden
    foods as fish, meat, spicy dishes, and wines,
    and engaging in sexual intercourse. The sole novelty was that these acts
    were to be undertaken\dots under the direction of a religious
    teacher, being regarded as concomitants of\dots (an) absolutely
    indispensable spiritual exercise.\supercite{heinrich:philosophies-india}\footnotemark
}

\footnotetext{Note especially the comments of Esteson and Wilson in chapter II, besides Blake's statements, for example, pp. \pageref{self:03} of this thesis. D. H. Lawrence has used the same idea, for instance, in \emph{The First Lady Chatterley}: \enquote{\dots inside nature there is a spark which sometimes flies into consciousness\dots as a result of the perfect contact} of opposites.\supercite{lawrence:first-lady-chatterly}}

\phantomsection
\label{self:18}

Elaborating further on the \enquote{metaphysical implications
of the corporeal spirituality} of the rite which has been
stigmatized by the psychologist J. H. Leuba as \enquote{the sexual
indulgences connected with the worship of certain non-civilized
and half-civilized peoples.}\supercite{leuba:god-or-man}
Zimmer says, \enquote{The basic Indian doctrine---the doctrine of transcendental monism,
which merges opposite principles in timeless union---finds no
more striking symbolization anywhere than in the lamasery
cult of the icon of the holy bliss (\emph{Mahāsukha}) of the
united couple.}\supercite{heinrich:philosophies-india}\footnote{It is widely believed\supercite{cheney:walked-with-god} that many of Blake's paintings (as well as books), were destroyed for their \enquote{immorality} (and \enquote{blasphemy}); it is possible that among these were mystical illustrations on the order of the Hindu sculpture of this period, for instance those at Khajuraho.}

\subsection{Taoism}

The preceding suggests, to the typical Western person,
the Taoist doctrine of the \enquote{Ying and Yang,} or \enquote{Yab-Yum,} that
is, the universal creative interplay of opposites;\footnote{Called \enquote{contraries} by Blake, as well as by the Taoists.}
although there was considerable contact between Taoism and Buddhism
during the first six centuries A.D., during which time Taoism
adopted many things from Buddhism, the Yab-Yum doctrine
seems to have originated independently in each place, though
probably earliest in China. A concise summary of the general
doctrine of philosophical Taoism, as well as an indication
of its historical classification, is given by the \emph{Encyclopedia
Britannica} as follows:

\quotebox[2.25em]{%
		The Taoists were mystics, but they were practical
		mystics, and hoped to realize the best social order through
		a harmonious relationship with the Tao. Their idea was
		\enquote{this worldly.} Their mysticism had three stages: (1) the
		purgation, casting out selfishness and self-seeking; (2)
		union with the Tao, by which the individual lost his individuality
		with the distraction of the contraries; (3) power,
		which enabled the individual merged with the Tao to escape
		the limitations of time and space.\supercite{britannica:encyclopedia}
}

Although this summary is worded in the standard Western \enquote{mystical}
terms, implying a conception of mysticism as necessitating
a metaphysic of ultimate dualism, its first two sentences
clearly indicate the intended sense of the \enquote{purgation,}
\enquote{individuality,} and \enquote{escape of time and space}; the material
immediately following reveals the \enquote{standard} Western position.\footnote{See pp. \pageref{self:04}-\pageref{self:05} of this thesis.}

\phantomsection
\label{self:04}

\section{Western Dualism (Christian)}

\subsection{Saints and Semi-Saints, or Emanations and Immanence}

According to the \emph{Encyclopedia of Philosophy and Religion},
the main ideas of Christian mysticism, both Catholic and
Protestant, are that \enquote{behind the visible, material, eternal universe,
which is the \emph{mother} of the one that we see,} and that it is
man's highest goal to apprehend that universe in some supernatural
fashion.\supercite{hastings:philosophy-religion}
In accordance with their dualism, the
Catholic mystics (Boehme will be shown to be typical of a
less dualistic\footnote{It must be noted that the doctrines of \enquote{Emanation} and \enquote{Immanence} are both dualistic philosophies, although the latter does approach monism to some degree. The dualistic distinction is maintained in the doctrine of Immanence by holding that it is \enquote{Being} which is \enquote{Immanent} within \enquote{Becoming.} Dante and St. Thomas Aquinas can be considered representatives of \enquote{Emanation} doctrine, while Plotinus and Boehme support the doctrine of \enquote{Emanation.}\supercite{underhill:mysticism}}
group, mainly Protestants), to some degree\supercite{underhill:mysticism}\footnote{See pp. \pageref{self:06} of this thesis.}
consider the apprehension of the \enquote{supernal reality} to be
dependent upon their \enquote{discarding} of the physical universe.

That for many Catholic mystics the \enquote{mystical experience}
is considered to be some sort of an approximation of death
is indicated by statements from, for instance, St. Bernard,
Meister Johannes Eckhart, and Dionysius, the Areopagite, and
the Orthodoxly accepted\footnote{As expressed by Evelyn Underhill.}
attitudes concerning events in the lives of, for instances, St. Catherine of Genoa and St. Catherine
of Siena. The simplest statement of the idea is probably
Eckhart's: while speaking to God, who appeared to him as a
naked and lovely boy, he asked where God was to be found, and
received the answer \enquote{in departure from everything worldly.}\supercite{cheney:walked-with-god}

St. Bernard, in a more negative statement, said \enquote{fasting,
praying, keeping watch, undergoing disciplines, wearing hair
shirts, sleeping on boards, etc., were all invented because
there is continual opposition of the flesh to the spirit,
the body threatens to overcome the spirit and there is unending
conflict between them.}\supercite{cheney:walked-with-god}
It seems to be a justifiable supposition that the rejection of consciousness by the mystics
increases proportionately with the degree of dualism of the
world-view; Dionysius, who, according to Cheney, tends to
minimize \enquote{the hole of the indwelling Christ,} said, \enquote{We must be
transported wholly out of ourselves and given unto God.},\supercite{underhill:mysticism}
and \enquote{You should, in the purposive practice of mystic contemplation,
escape the senses and lay aside the guidance of the
intellect \dots escaping alike what is and what is not\dots}\supercite{cheney:walked-with-god}

\subsection{Modern Scholars of Christian Mysticism (Dualistic)}

The extent of Dionysius' dualism is far beyond that of the
common \enquote{being-becoming} dichotomy: not only is all of that
which forms the common denominator of the \enquote{being-becoming}
dichotomy, viz., existence, rejected, but a category of \enquote{non-existence}
is oriented, and it is rejected also; that which
constitutes the mystic's goal is so utterly removed from the
ordinary world that it is unknowable, even by the soul. The
use of negative statements and even paradox is reminiscent
of Lao Tzu's statement, \enquote{He who knows does not speak; he who
speaks does not know. Soften its light, submerge its turmoil---this
is the mystic unity.}\supercite{browne:great-scriptures}
Dionysius, however, is more extreme than the Taoists; for instance, in his statement
\enquote{\dots rise upward
toward union with him who is above all knowing
and all being,}\supercite{cheney:walked-with-god}
ineffability is replaced by unknowability, unless
\enquote{the essential mystical darkness, the cloud of unknowing}
indicates merely the absence of verbal knowledge,
which seems extremely doubtful. It is interesting to note
that both of these negative positions, that which clearly
advocates a monistic following of the \enquote{way,} and that which
urges that the soul leave behind physical existence in its
attempted apprehension of the \enquote{divine darkness,} resemble
each other in the denial of existence in the \enquote{highest} (nothing),
(the Tao, and God, respectively), while it is merely
the proper \enquote{goal} of the mystic in which they apparently
differ; that is, they could be said to have the same metaphysics,
while differing in their ethos.

Underhill indicates that the lives of certain saints
reveal the inverse relationship between physical and spiritual
well-being: \enquote{\dots in cases of St. Catherine of Genoa
and St. Catherine of Siena it would seem that as their health
became feebler and the nervous instability always found in
persons of genius increased, their ecstasies became more
frequent\dots}\supercite{underhill:mysticism}
Underhill also quotes St. Thomas on this subject,
saying: \enquote{St. Thomas proves ecstasies (trances) to be
inevitable\dots \enquote*{The higher our mind is raised to the contemplation
of spiritual things,} he says, \enquote*{The more it is abstracted from sensible things.
(But the final term to which contemplation can possibly arrive is the divine substance.)
Therefore, the mind that sees the divine substance must be
totally divorced from the bodily senses, either by death or
\emph{by some rapture}.}}\supercite{underhill:mysticism}
Cheney, a less dualistic commentator on
the mystics, rather than exaulting the state of trance or
near death as the ultimate form of enlightment, says that
those mystics, exemplified by St. Catherine of Siena, were
extremists, \enquote{\dots given to penance and ecstatic visions and
trances\dots}\supercite{cheney:walked-with-god}

Although it is not entirely approved by the dualistic
writers, a physically produced, and sensuously conscious,
\enquote{ecstasy} is not considered by them to be wholly without
value: it seems that the concept of \enquote{mono-ideism} is related
in their minds to spirituality, since, \enquote{In the mystic, the
idea which fills his life is so great a one---the idea of God---that,
in proportion as it is vivid, real, and intimate, it
inevitably tends to monopolize the field of consciousness}\supercite{underhill:mysticism}
with the result that any experience that seems to be mono-ideistic,
or to result from mono-ideism, is considered to be
\enquote{spiritual} to a degree, regardless of its physical origin
and apparently somewhat sensuous nature, since (by their
logic) to exclude sensations numerically is to approach
the transcendent reality more closely.

Thus Boehme, and those others such as St. Ignatius Lovola
whose \enquote{mental eyes} were opened to a superior understanding
by contemplation of a physical object,\footnote{Underhill considers Blake to be a member of this group of \enquote{imperfectly dualistic} mystics; see pp. \pageref{self:07} of this thesis.}
are considered to be mystics of a moderate state of advancement, since it seems that
they, in the dualists language, overcame sensuous perception
to a degree which enabled them to see the spiritual nature of
reality, although it was somewhat contaminated by the remaining
awareness of particular objects. Boehme reveals the dual
nature of the \enquote{union} as he understood it, in which the
finite entity retains its nature while being \enquote{defied} by the
presence of God, in statements such as the following:
\enquote{\dots if thou art born in God, then there is in thyself (in the
circle of thy life) the whole heart of God undivided}\supercite{underhill:mysticism}
and \enquote{\dots the soul (is) set in the deity; the deity penetrateth
through the soul, and dwelleth in the soul, yet the soul doth
not alter it (from being a soul) but only giveth it the
divine source (or property) of the majesty.}\supercite{underhill:mysticism}
Obviously this is a less dualistic general world-view than that of the
mystics who hold the doctrine of emanations, since matter
and spirit are not held to be in absolute opposition, it is
this tendency toward monism that makes Boehme's conception of
unity insufficiently pure and separate for the dualists,
while it is his failure to philosophically remove the distinction
between body, soul, and God, or the \enquote{oversoul,} that
makes his doctrine not entirely acceptable to the \enquote{monistic
mystic}; nevertheless, many of his mystical statements,
that is, his descriptions of the mystical experience, have
been of great value to many such \enquote{monistic mystics,} an
example of such a statement would be his statement that an
intensely perceived object led him to the ability to see
\enquote{the principles and deepest foundations of things.}\supercite{underhill:mysticism}
It is interesting to note that this idea appears also in the descriptions
of the first mystical experiences of John Fon, and others.

Evelyn Underhill, in \emph{Mysticism}, seems to be one of the
most \enquote{valuable} writers on mysticism, because of the fullness
of her statement of the metaphysic which underlies her interpretation
of the experiences of the various subjects considered.
Although she is very definite in her labelling
of other attitudes as wrong, though the \enquote{other enemy} varies,
she at least treats her subject seriously enough to recognize
that there are alternative possible interpretations of the
meaning of \enquote{mysticism,} the definition of mysticism which
she defends in this book is given in the preface as follows:

\clearpage

\quotebox[2.25em]{%
    Broadly speaking, I understand it to be the expression
    of the innate tendency of the human spirit towards complete
    harmony with the transcendental order; whatever be the theological
    formula under which what order is understood. This
    tendency, in great mystics, gradually captures the whole
    field of consciousness; it dominates their life and, in the
    experience called \enquote{mystic union,} attains its end. Whether
    that end be called the God of Christianity, the World-Soul of
    Pantheism, the absolute of philosophy, the desire to attain it
    and the movement towards it---so long as this is a genuine
    life process and not an intellectual speculation---is the
    proper subject of mysticism.\supercite{underhill:mysticism}
}

This generally Neo-Platonic definition of mysticism, said to
be \enquote{\dots its old meaning\dots the science or art of the spiritual
life}\supercite{underhill:mysticism}
is opposed to such \enquote{abuses} of the word as its use
\enquote{\dots as an excuse for every kind of occultism, for dilute
transcendentalism, vapid symbolism, religious or aesthetic
sentimentality, and bad metaphysics.}\supercite{underhill:mysticism}
Apparently, \enquote{every kind of occultism} is intended to include the oldest use of
the word, i.e., the Greek use of it mentioned at the beginning
of this chapter.

\phantomsection
\label{self:12}

Since the above definition depends upon the meaning
\enquote{spiritual}, Miss Underhill's definition of that must be
shown; it clearly cannot be \enquote{naturalistic} in any sense, since
she has said, in the preface to the twelfth edition, \enquote{Determination---more
and more abandoned by its old friends the physicists---is
no longer the chief enemy to a spiritual interpretation
of life and is required by the experience of the
mystics, it is rather a naturalistic monism, a shallow
doctrine of immanence unbalanced by any adequate sense of
transcendence, which now threatens to re-model theology in
a sense which leaves no room for the noblest and purest
reaches of the spiritual life.}\supercite{underhill:mysticism}
That the above definition was purely in the Neo-Platonic tradition is indicated by
her statement that she has consistently believed \enquote{\dots that
the facts of man's spiritual experience pointed to a limited
dualism; a diagram which found place for his contrasting
apprehension of absolute and contingent, being and becoming,
simultaneous and successive. Further, that these facts
involved the existence in him, too, of a certain doubleness,
a higher and lower, natural and transcendental self\dots}\supercite{underhill:mysticism}
This \enquote{limited dualism} is apparently \enquote{limited} only in the
way that the dualism of Plotinus is limited, viz., it is not
one which says that people are entirely alien to the transcendent
reality, which would obviate the possibility of a
mystical experience, but one which is entirely supernaturalist,
except that in the association of a \enquote{soul} with a body
there is some sort of a graduation of reality. The \enquote{soul,} to
Underhill, is apparently stretched between the \enquote{natural and
transcendental} selves, since it is only the \enquote{apex of the
soul\dots which the mystics have always insisted} to be \enquote{the
instrument of their special experience.}\supercite{underhill:mysticism}
Disregarding the degree to which the dualism is limited, it is worthwhile to
note the way in which Underhill elaborates upon the subject
of dualism in relation to mysticism: \enquote{this reinstatement of
the transcendent, the \enquote*{wholly other,} as \emph{the} religious fact,
is perhaps the most fundamental of the philosophic changes
which have directly affected the study of mysticism.}\supercite{underhill:mysticism}

Closely connected with the transcendence of its (mysticism's)
object, are the following two doctrines: \enquote{First that mysticism\dots can
never be the whole content of\dots religion. It requires
to be embodied in some degree in history, dogma, and
institutions\dots secondly, that the antithesis between the
religions of \enquote*{authority} and \enquote*{spirit,} the \enquote*{church} and
the \enquote*{mystic} is false.}\supercite{underhill:mysticism}
Since nothing in the assumption of a \enquote*{transcendent object} leads necessarily to these doctrines,
it can be assumed at this point that a favorable treatment
will be given to those mystics and remained within the
church. while a less favorable treatment will be
given the less Orthodox individuals. This is clearly indicated
by Miss Underhill's statement that \enquote{The \enquote*{exclusive} mystic,
who condemns all outward forms and rejects the support
of the religious complex, is an abnormality. He inevitably
tends towards Pantheism, and seldom exhibits in its richness
the unitive life.}\supercite{underhill:mysticism}

In explaining the \enquote{characteristics of mysticism} Underhill
gives four \enquote{rules} which can be used to \enquote{test} the
validity of cases which claim \enquote{to rank among the mystics};
they are intended especially to be reputations of two of
William James' \enquote{four harms}\footnote{Those who use the word in what seems to be its \enquote{true,} or earliest meaning, i.e., in its radical signification, would also deny that the experience was ineffable and transient.}
of the mystic state, namely, \enquote{noetic quality} and \enquote{passivity.} The first \enquote{rule} is intended
to distinguish the mystical experience from the
simply mystical philosophy, I.E., Platonism: \enquote{1. True
mysticism is active and practical, not passive and theoretical.}
In developing this idea Underhill emphasized this well-known
distinction between the philosophies of Plato and
Plotinus, viz., that Plato's \enquote{unity} was only an intellectual
thing, an knowledge of the \enquote{truth,} while that of Plotinus
was an experiential thing, a \enquote{flight of the alone to the alone.}\supercite{turnbull:essence-of-plotinus}
Plotinus is said to be one of those (Platonic
philosophers) \enquote{\dots who have passed far beyond the limits of
their own philosophy, and abandoned the making of diagrams
for an experience, however imperfect, of the reality at which
these diagrams hint.}\supercite{underhill:mysticism}
Platonism, says Underhill, \enquote{\dots is the reaction of the intellectualist upon mystical truth}\supercite{underhill:mysticism}
and it is implied that in Plato's case the \enquote{mystic truth} was from
a source other than himself.

The second \enquote{rule} is one which is very important to note
for its implications, concerning ethics, which contrast so
sharply with the statements made by or about the \enquote{mystics} of
the \enquote{naturalistic} sort:

\phantomsection
\label{self:06}

\quotebox[2.25em]{%
    Its (mysticism's) aims are wholly transcendental and
    spiritual. It is in no way concerned with adding to, exploring,
    re-arranging, or improving anything in the visible
    universe. The mystic brushes aside that universe, even in
    its supernormal manifestations.\supercite{underhill:mysticism}
}

The third \enquote{rule} seems to be merely a slight variation
of the first: the changeless \enquote{\dots one is for the mystic, not
merely the reality of all that is, but also a living and
personal object of love; never an object of exploration.}\supercite{underhill:mysticism}

The fourth, however, offers some interesting facts concerning
the \enquote{definite psychological experience} which is entailed
by mysticism: the experience, which is sometimes called
\enquote{ecstasy,} though Underhill prefers the words \enquote{unitive state,}
or the \enquote{mystic life process,} is defined only by its prerequisites,
which are \enquote{the apprehension of God,} \enquote{the passion
for the absolute,} \enquote{an appropraite psychological make-up,}
\enquote{a nature capable of extraordinary concentration, an exalted
moral emotion,} and \enquote{a nervous organization of the artistic type.}\supercite{underhill:mysticism}

Sheldon Cheney, in \emph{Men Who Have Walked with God},\supercite{cheney:walked-with-god}
gives a definition of Christian mysticism which, though it
applies to several famous mystics to some degree, seems to
be determined by a consideration of Blake's mystical life;
it is distinctly not the definition that would be given by
a writer of the type of Evelyn Underhill, although that
type would agree with his statement that Oriental mysticism
is \enquote{negative.} \enquote{The Christian mystic,} Cheney says in defining
the motivation which apparently is for him the explanation
of the \enquote{Blakean} character of Christian mysticism,\footnote{That Cheney is clearly in error when he thus opposes Christian to \enquote{Oriental}---later specified as Buddhist---mysticism is seen when one notes the great stress given by the Buddhists to the \enquote{motivation}---compassion (Karuna)---that \enquote{withholds the Bodhisattva from Nirvana.}\supercite{heinrich:philosophies-india}}
\enquote{while losing nothing of the sublimity of the abstract union with the absolute\footnote{See the footnote on pp. \pageref{self:08} of this thesis; this though a distinctly non-Blakean idea, is consonant with Cheney's interpretation of Blake.}
as known to Eastern sages, is likely to substitute a \enquote*{contemplation of the heart}---Bernard's phrase---for
intellectual meditiation.}\supercite{cheney:walked-with-god}
\enquote{The Christian founders substituted, in place of the abstract one, a sympathetic God-Father.}\supercite{cheney:walked-with-god}

As opposed to the motivation, Cheney describes the
character, or what might be called the \enquote{ethic,} of Christian
mysticism: \enquote{Typically, the Christian mystic curbs the inclination
to seclusion.} \enquote{\dots he follows contemplation with
service, abstention with participation in active works.}\supercite{cheney:walked-with-god}
\enquote{\dots from the first mystics among the Apostles to that latest
Christian mystic poet (Blake) whose hand never rested from
\enquote*{my endeavour to restore the Golden Age,} to restore the age
when man finds \enquote*{Eternity in an hour}---from first to last the
great Christian mystics stayed out their lifetimes in the
current of mortal occupations\dots they have returned\dots to
illumine that corner of the Earth about them, or it may be
a whole nation or realm, with light from their vision and
their understanding.}\supercite{cheney:walked-with-god}

\enquote{\dots Christian mysticism implies less retreat from the
world, a withdrawal into the light of the Divine, than an
enlargement of the mortal horizon and a mission among men to
reveal to them the joy of knowing eternal life in the midst
of mortal affairs.}\supercite{cheney:walked-with-god}
Cheney shows Buddhism to be a religion that induces \enquote{an admirable social ethic, but only as incidental
along a path of personal mystic experience.} He says that
the end of that path is \enquote{\dots Nirvana, or extinction of selfhood
in the ocean of eternal divinity.}\footnote{Etymologically, \enquote{\emph{Nirvana}} can be considered to mean simply \enquote{without the forest,} with, possible, the implication of the colloquial English, \enquote{out of the woods}: \emph{nir}, \enquote{without,} and \emph{vana}, \enquote{the forest.}\supercite{heinrich:philosophies-india}}

\section{Technical Studies of Mysticism}

\subsection{Philosophical}

Cheney continues the discussion of the relation of Buddhism
to Christianity with the somewhat Blakean statement, \enquote{There is a negative aspect to the Buddhist faith, a denial
of the\dots importance of life in the world, which is fundamentally
different from the message that can be read in the words, and
in the life of Jesus.}\supercite{cheney:walked-with-god}
That the difference is mainly one degree is indicated by his summary: \enquote{Nevertheless the
Christian faith advances a way of life not unlike the Buddhist
in\dots its positing of divine immersion or communion as
the highest good in mortal life.}\supercite{cheney:walked-with-god}

It is in accord with the incompleteness of Cheney's
dualism that he is concerned with the mystical experience
as a means to the end of social well-being.\supercite{cheney:walked-with-god}

Alfred Kazin's discussion of Christian mysticism\supercite{kazin:portable-blake}
should be especially valuable to this chapter, since it seems to be
written with at least a fair amount of objectivity, while
Kazin is a very well-known Blakean scholar. Christian mysticism, he says:

\quotebox[2.25em]{%
    \dots is founded on dualism. It is rooted in the belief
    that man is a battleground between the spirit and the flesh,
    between the temptations of Earth and God as the highest God.
    The mystic way is the logical and extreme manifestation of
    the spiritual will, obedient to a faith in supernatural
    authority, to throw off the body and find an ultimate release in
    the Godhead. Christian mysticism is based upon a
    mortification of the body so absolute that it attains a condition
    of ecstasy. To the mystic, God is the nucleus of the
    creation, and man in his Earthly life is a dislodged atom
    that must find its way back. The mystic begins with submission
    to a divine order, which he accepts with such conviction
    that Earthly life becomes nothing to him. He lives
    only for the journey of the soul that will take him away,
    upward to God. What would be physical pain to others, to
    him is purgation.
}

\phantomsection
\label{self:05}

Of course, there are several weaknesses in this sort of
generalization, including the omission of definitions for
such terms as \enquote{ecstasy,} especially with reference to the
different Christian beliefs concerning that doctrine, and
the implication that there is only one, simple, degree of
dualism, besides the necessary neglect of even the most important
Non-Christian mysticisms, but it does serve the important function
of providing a fairly objective, over-all view of the subject
by one who is, apparently, an \enquote{outsider.}

\phantomsection
\label{self:15}

A nineteenth century psychologist, Ernst Mach, in writing
an epistemology for scientists\supercite{boring:experimental-psychology}
(the first chapter of which is called \enquote{Antimetaphysical}), used the \enquote{facts} of the
\enquote{liberal} mystical tradition, viz., elimination of the \enquote{theoretical} self,
but not of the sense-consciousness, affirmation of dreams as
\enquote{valid knowledge,} and denial of the Platonists' \enquote{realm of eternal ideal existences,}
but seems not to have used them as the basis for a \enquote{psychotherapy}
(with a goal of higher, more intense consciousness, rather than
of \enquote{normality}), as the Buddhists, for instance, have done. Edwin G. Boring,\supercite{boring:experimental-psychology}
in explaining Mach's ideas of consciousness and the world, said \enquote{sensations are not observed; they are given. Being given,
they cannot be shown to be in error. Illusions are \enquote*{illusory}: there are none,
or rather, the straight rod thrust into water is sent, and, if there
be any illusion, it is that the rod is still straight. There is no ego;
there are only sensory data. If we say \enquote*{it lightens,} we ought also to say, \enquote*{it thinks};
\emph{cogitat}, not \emph{cogito}. \enquote*{The world consists only of our sensations.}
Dreams are as valid knowledge as perception.}\supercite{boring:experimental-psychology}
All that is eliminated in this denial of the existence of \enquote{ego} is
the abstract conception of consciousness: consciousness is
here considered to be an \enquote{unbounded} system of sensations, and
the tendency to give conscious value to assumptions of
non-temporal existences, either \enquote{outside} (\enquote{the world consists\dots}, etc.),
or \enquote{inside} (\enquote{There is no ego\dots}), is
rejected, and, of course, this includes the Platonic idea
of \enquote{eternal ideal existences} and the dualists' belief in
a soul distinct from the body.\footnote{See chapter III, second section, of this thesis.}

A very different attitude will be discussed next, as a
contrast for the purpose of showing that an opposition exists
between the facts of monistic mysticism and those of dualistic
mysticism; and that this opposition might be based on different
attitudes towards symbols, symbols being \enquote{things} the
peculiar nature of which is ignored, in favor of their
\enquote{meaning,} that is, another \enquote{thing} which is, by a mental
process, associated with the first.

An attempt to define \enquote{mysticism} philosophically has
been made by a well-known contemporary philosopher, Charles
Morris, in an essay called \enquote{Comments on Mysticism and its Language,}\supercite{hayakawa:language-meaning-maturity}
in which he explains his belief that the concept of language---based on
mental interpretation \enquote{\dots is essential to the understanding of art,
myths, magic, the totem, religion, prestige, race prejudice, and the complex types of perception}\supercite{hayakawa:language-meaning-maturity}
as well as mysticism.

Morris' explanation of language as the basis of mysticism
rests upon his idea of \enquote{the role-taking function of language}:
it is his belief that by means of \enquote{\dots language one can symbolize times
and places other than the here and now, and persons and things
other than the speaker himself}\supercite{hayakawa:language-meaning-maturity}
and, in a sense, \enquote{become} the things which are \enquote{signified.} The point
to be stressed, he says, \enquote{\dots is that in this socially derived
process of role-taking one can become symbolically an object
other than the self of the here and now\dots}\supercite{hayakawa:language-meaning-maturity}\footnote{The fallacy of these ideas should be obvious: a thing which \emph{means} \enquote{far away}or \enquote{near by} is a different thing from human conscious existence in a certain location, and even if possession of a symbol were equivalent to possession of the existence represented by the symbol, the quality of subjective consciousness must, by definition, be described (when referring to its location, without \enquote{objectifying} it, or describing it relationally) by the word \enquote{here}; \enquote{note here,} as Morris fails to see, is purely a relational statement, giving the location of an object which is extraneous to the conscious speaker. It can be seen, therefore, that the most extreme modification of consciousness that could be obtained by this method would be the possession of the \enquote{existence represented by the symbol} (which is denied above), the possession of an \enquote{existent} environment made up by the \enquote{existences} \enquote{carried within} the symbols. Of course, if symbols are considered to be transcendently valuable, the shift of consciousness from sensuous reality to a \enquote{world of symbols} might be considered to be sufficiently valuable to deserve the name \enquote{mystic experience.}}
According to his theory, \enquote{\dots this simultaneous, or nearly simultaneous arousal
of the complex and often contradictory role-taking processes made
possible by language constitutes an essential part of the mystical
experience.}\supercite{hayakawa:language-meaning-maturity}

The experience of seeing ordinary objects \enquote{\dots through
symbolic eyes enlarged by cosmic wandering} is \enquote{liberating,}
according to Morris; \enquote{Einstein has testified to this, and has
even spoken of it as \enquote*{the sower of all true art and science.}}
Whether the experience is liberating or not depends upon one's
conception of \enquote{freedom,} but the rest of Morris' statement is
decidedly false: what Einstein spoke of as the \enquote{mystical feeling,} the
\enquote{sower\dots} etc., was simply a non-dogmatic perception of the universe, or reality.
That his attitude was entirely contradictory to that ascribed to him by Morris is indicated
by his own analysis of his mental activity:\supercite{ghiselin:creative-process}
\enquote{The words or the language, as they are written or spoken,
do not seem to play any role in my mechanism of thought.}
\enquote{Conventional words or other signs have to be sought for laboriously only
in a secondary stage\dots} \enquote{In a stage when words intervene at all\dots they intervene only in a secondary stage\dots}\supercite{ghiselin:creative-process}
Morris also claims that \enquote{The psychologist, A. H. Maslow,
has found it (the \enquote*{symbolic experience}) to be present in some degree
in persons of maximum creativity and \enquote*{psychological health,}} and that it is \enquote{\dots available in varying degrees
to all persons, regardless of their scientific and philosophical commitments.}\supercite{hayakawa:language-meaning-maturity}
The truth in this case is that Maslow clearly and explicitly said that the \enquote{mystical} consciousness
appears only in those individuals who have been \enquote{liberated} from the effects of
language and symbols, that the brain-injured and neurotics are typically limited
to \enquote{symbolic} understanding of reality, and that \enquote{philosophical commitments}
are integrally related to the frequency and intensity, and even to the very existence
of the mystical experience. These attitudes will be more fully discussed in the
following section on Maslow, and an alternate form, viz.,
that the \enquote{mystical experience} may have been artificial (or \enquote{chemical}), rather than
an exclusively natural (or \enquote{psychological}) origin, as represented by Aldous Huxley, will then
be discussed, while the relation between \enquote{philosophical commitments} and \enquote{natural sources}
will be considered again in chapter III.

\phantomsection
\label{self:29}

\subsection{Psychological}

A. H. Maslow, a psychologist who has spent much time in the
study of what he considers to be people in an extremely rare condition
of complete psychological health, says that for these subjects, \enquote{Those
subjective expressions that have been called the mystic experience and described so well by
William James are a fairly common experience\dots}\supercite{maslow:motivation-personality}
He says that these experiences are related to the strong, free emotions which
are typical of his subjects. Reminiscent of the famous ideas of
\enquote{mystic marriage} or \enquote{sacred betrothal,}\footnote{See St. Teresa's \enquote{spiritual marriage,}\supercite{underhill:mysticism}}
but with the differences that his subjects are speaking from a point of view
opposite to that of the famous mystics, Maslow says, \enquote{My interest and attention
in this subject (mysticism) was first enlisted by several of my subjects who described their
sexual orgasms in vaguely familiar terms which later I remembered had been used by
various writers to describe what they called the mystic experience.} More specifically,
he says, \enquote{There were the same feelings of limitless horizons opening up to the vision,
the feeling of being simultaneously more powerful and also more helpless than one ever
was before, the feeling of great ecstasy and wonder and awe, the loss of placing
in time and space with, finally, the conviction that something extremely important and
valuable had happened, so that the subject is to some extent transformed and
strengthened even in his daily life by such experience.}\supercite{maslow:motivation-personality}

\clearpage

Revealing himself to be, in general, allied with those
called by Underhill \enquote{monists and philosophic naturalists,}\footnote{E.g., Leuba, who believes that \enquote{life, more life, a larger, richer, more satisfying life, is in the last analysis the end of religion.}\supercite{oxford:the-monist} Colin Wilson's position (see chapter II), is practically identical with this.}
Maslow says, \enquote{It is quite important to dissociate this experience
from any theological or supernatural reference, even
though for thousands of years they have been linked. None
of our subjects spontaneously made any such tie-up.}\supercite{maslow:motivation-personality}
He suggests that renaming it \enquote{the oceanic feeling} (Freud's term for it)
would help to remove any implication of the supernatural from it.

Another attitude toward the mystical experience that seems to be
common among the \enquote{monists}\footnote{See the comments by Huxley on pp. \pageref{self:09}-\pageref{self:10} of this thesis.}
is that it can occur in varying intensities: \enquote{The theological literature has
generally assumed an absolute, qualitive difference between the
mystic experience and all others. As soon as it is divorced from
supernatural reference and studied as a natural phenomenon,
it becomes possible to place the mystic experience
on a quantitative continuum from intense to mild.}\supercite{maslow:motivation-personality}
On the basis of this attitude he indicates that the \enquote{\emph{mild} mystic experience}
occurs in many or even most persons, and that a \enquote{favored individual}
will experience it many times a day.\supercite{maslow:motivation-personality}
The acute mystical experience is, he says, \enquote{\dots a tremendous intensification of \emph{any} experiences in which there
is a loss of self of transcendence of it,}\supercite{maslow:motivation-personality}
for instance, intense sensuous experience. \enquote{It may even be that the so-called mystic experience is the perfect and extreme
expression of\dots full appreciation of all the characteristics of the particular phenomenon.}\supercite{maslow:motivation-personality}

\phantomsection
\label{self:30}

Maslow believes, as Huxley does, that language (and its associated abstraction and associative reasoning)
limits the experiences of the \enquote{oceanic feeling}: \enquote{\dots it is a screen
between reality and the human being.}\supercite{maslow:motivation-personality}
\enquote{It is\dots very obviously and frankly a means\dots for dulling the perceptions\dots}\supercite{maslow:motivation-personality}
To overcome the limiting effects of language and an education which is
concerned mainly with memory (\enquote{far too much occupied with the intellectual analysis}),\supercite{whitehead:modes-of-thought}
Maslow says that it is necessary to be concerned with \enquote{\dots fresh experiences, with concrete and particular realities.}\supercite{maslow:motivation-personality}

As Cheney and Huxley do, Maslow believes that the mystical sort of consciousness
can have tremendous effects on society: he describes a society constituted entirely of
people who resemble his \enquote{subjects} and gives his opinion that it would be the best society possible.
In this society \enquote{\dots the deepest layers of the human nature could show themselves with great ease.}\supercite{maslow:motivation-personality}

Finally, the interrelations of \enquote{intellectual intoxication,}
which, it seems, might also be called \enquote{biological,} \enquote{psychological,}
or \enquote{sensory} intoxication, with certain kinds of chemical intoxication,
and their contrasts to symbolic, dualistic, and \enquote{transcendent} \enquote{mystical
experiences,} will be illustrated by a discussion of the entire
subject by another \enquote{monistic-naturalistic} mystic, Aldous Huxley.

\phantomsection
\label{self:09}

Aldous Huxley, who has shown a Blakean influence in his
writings for several decades (e.g., \emph{The Cicadas, and Other Poems}), gives
in \emph{The Doors of Perception} a description of his
personal chemically induced \enquote{mystical experience,} which, he
is convinced, resembles the experiences of Blake, whose
\enquote{mental species,} he believes, \enquote{\dots is fairly widely distributed
even in the urban-industrial societies of the present day.}\supercite{huxley:doors-of-perception}
Although Huxley believes himself to be a poor subject for the experiment
(which began with the taking of four-tenths of a gram of mescalin, the synthetically produced
form of the drug which is found naturally in the Aztec's plant-god \emph{Peyotl},
and which is biochemically similar to that found in the other
Aztec plant-god, \emph{Teonanacatl}), saying that he had \enquote{always been
a poor visualizer,} and that his mental images \enquote{have little
substance and absolutely no autonomous life of their own,}
the intensity of the experience was apparently sufficient to
cause him to alter some of his theories of religion and metaphysics
toward a more complete agreement with those of Blake.\supercite{huxley:doors-of-perception}

Huxley describes the mystical experience both indirectly
and directly, that is, he limits the field of mysticism by
discrediting dualism, and he describes the experience of
seeing \enquote{infinity} in the world of material objects. Among
his somewhat negative demonstrations of the nature of mysticism
is a comment on Plato:

\quotebox[2.25em]{%
    \dots Plato seems to have made the enormous, the grotesque mistake
    of separating being from becoming and identifying it with
    the mathematical abstraction of the idea. He could never,
    poor fellow, have seen a bunch of flowers shining with their
    own inner light and all but quivering under the pressure of
    the significance with which they were charged; could never
    have perceived that what rose and iris and carnation so
    intensely signified was nothing more, and nothing less,
    than what they were---a transience that was yet eternal life\dots\supercite{huxley:doors-of-perception}
}

Thus Huxley uses a sort of Platonic language to completely
deny Plato's main doctrine, i.e., that there is a higher order
of being which is perfect and unchanging, which was the basis
for most of the subsequent Western philosophy and, therefore,
of Western mysticism until the rise of Western pantheism and
pantheistic mysticism came with such philosophers as Benedict Spinoza.

It is because of his perception of a long tradition of
erroneous philosophy and false mysticism in the West that
Huxley turns, in this book, to the Orient for most of the
examples of historic parallels to his experience. This attitude
is summed up in the statement:

\quotebox[2.25em]{%
    In their art no less than in their religion, the Taoists
    and the Zen Buddhists looked\dots through the Void at \enquote{the ten
    thousand things} of objective reality. Because of their doctrine
    of the Word made flesh, Christians should have been
    able, from the first, to adopt a similar attitude towards
    the universe around them. But because of the doctrine of
    the Fall, they found it very hard to do so. As recently as
    three hundred years ago an expression of thorough-going
    world denial and even world condemnation was both orthodox
    and comprehensible.\supercite{huxley:doors-of-perception}
}

In a more positive mood, Huxley reveals what must be done to achieve
the \enquote{real} mystic consciousness, or, it could be said, what must be done to overcome the
false perception which is associated with a dualistic attitude it is simply
that \enquote{\dots we must preserve and, if necessary, intensify our
ability to look at the world directly and not through that
half opaque medium of concepts, which distorts every given fact into the
all too familiar likeness of some generic label or explanatory abstraction.}\supercite{huxley:doors-of-perception}

\phantomsection
\label{self:10}

Concerning his \enquote{looking at the world directly,} Huxley
says, \enquote{\dots now I know contemplation at its height.}\supercite{huxley:doors-of-perception}
and \enquote{my actual experience had been, was still, of an indefinite
duration or alternatively of a perpetual present\footnotemark
made up of one continually changing apocalypse.}\supercite{huxley:doors-of-perception}
Again, while looking at the flowers (mentioned above in connection with his criticism of Plato)
which shone \enquote{with their own inner light,}
he was conscious of them as \enquote{\dots a bundle of minute, unique particulars\footnotemark
in which\dots was to be seen the divine source of all existence,}\supercite{huxley:doors-of-perception}
and said, concerning \enquote{The Beatific Vision, \emph{Sat Chit Ananda}, Being-Awareness-Bliss,} \enquote{for the first
time I understood, not on the verbal level, not by inchoate hints or
at a distance, but precisely and completely what those
prodigious syllables referred to.}\supercite{huxley:doors-of-perception}
Although some people in approximately the same circumstances will experience extra-sensory
perceptions, and some will \enquote{discover a world of visionary beauty:} (he says, concerning this
fact, \enquote{like mescalin takers, many mystics perceive supernaturally brilliant colors,
not only with the inward eye, but even in the objective world around them.}),\supercite{huxley:doors-of-perception}
the more intense forms of experience as described by him seem to be of the most valuable for an understanding of mysticism. These are, first, the experience
of \enquote{\dots the glory, the infinite value and meaningfulness of naked existence, of the given, unconceptualized events,}\supercite{huxley:doors-of-perception}
and, secondly, \enquote{the final stage,} in which it is known \enquote{that
all is in all---that all is actually each.}
This is as near, I take it, as a finite mind can ever come to \enquote{perceiving everything that is
happening everywhere in the universe.}\supercite{huxley:doors-of-perception}
These two statements seem to imply, respectively monism and a sort of determinism;
the last sentence of that quotation \enquote{\dots a finite mind,} however,
seems to contradict the meaning of the preceding statement
that \enquote{all is in all,} that is, that each thing, is,
truly, infinite in some way. It seems obvious that Huxley is, in
effect, two persons, one who, under the influence of mescalin, can
experience and apparently be convinced of, certain things, and
the other, his ordinary self, who can remember the experiences
of the other self well enough to describe them vividly, but who is not
completely convinced of the meaning contained in them. Although he is aware of
this dual personality (except in an instance such as that
above, which involves the intellectual understanding of the world in
general, and which was revealed in Huxley by an
unintentional self-contradiction) of the mescalin taker,
Huxley is convinced that the chemically induced \enquote{mystical experience} causes
a permanent, and beneficial, change in the ethical beliefs
and functioning of the subject.\supercite{huxley:doors-of-perception}

\footnotetext{Note the striking similarity to Blake's comment, \enquote{a vision of the eternal now}\supercite{keynes:william-blake} concerning Lavater's aphorism: \enquote{whatever is visible is the vessel\dots of the invisible past, present, future---as man penetrates to this more, or perceives it less, he raises or depresses his dignity of being.} See pp. \pageref{self:11} of this thesis.}

\footnotetext{See pp. [illegible] of this thesis.}