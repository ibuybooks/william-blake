\chapter[Analysis of the Scholarship Which Considers Blake's Mysticism]{Analysis of the Scholarship Which Considers Blake's Mysticism}

\section{Categorization}

The writings which consider Blake as a \enquote{mystic} can be
divided, as \enquote{philosophizing} in general can, into the categories of (1)
emotionally treated discussions of what is
apparently a conventional association between the object
(Blake's works) and the label (\enquote{mystic}), which discussions
offer nothing more than information concerning the author,
(2) mainly rational applications of standard definitions
(either \enquote{colloquial} or \enquote{scholarly}) to the object, which
are valuable only if a highly specific (\enquote{scholarly}) definition
is shown to be positively applicable, since to \enquote{define} an
object by indicating that it is outside a very limited
class is much less accurate than the inexact method of
indicating that it is within a very broad class (the
\enquote{colloquial definition,} which in the case of \enquote{mysticism} is
especially broad because of its confusion with \enquote{mystery,} that
is, anything unfamiliar), and (3) considerations of the
given (perceptual) realities of the object, with the application
of a label which has approximately accurate etymological
implications, and from which all irrelevant meanings have
been eliminated. In general, studies which were published
before 1915 can be included in the first category, since the
study of mysticism seems to have been of a very vague nature
up to that time, and in the second category can be included
those studies which class Blake as a \enquote{mystic,} a \enquote{non-mystic,}
or \enquote{an example of psychic principle} (e.g., \enquote{sublimation}),
either entirely without definition of the term or with a
definition which has no discoverable reference to observable
reality, that is, which is not applicable to Blake.\footnote{It is thus suggested that the Blake studies in the second category are without value, since it is not granted that any \enquote{scholarly} (as opposed to \enquote{colloquial}) definition is \enquote{positively applicable} to Blake's works.}
The fact that the third category is used when the second category
contains the possibility of a satisfactory explanation of Blake
implies that there is no \enquote{standard} definition which
can be applied correctly to Blake's position; only one of this
group uses the word \enquote{mysticism} in describing Blake, and in
that case the word is apparently used in that \enquote{unusual,}
\enquote{functional} rather than supernatural, sense used by the Greek
mysteries and Maslow, and applied to Buddhism and other
socially oriented \enquote{cults.}

\section{First Category}

Since the first group has been designated as \enquote{unrelated,}
it will be sufficient to mention the names of typical authors
and their books, with comments where necessary.

Adaline Butterworth's \emph{William Blake, Mystic}\supercite{butterworth:blake-mystic}
is an ultimate example of irrelevant, though enthusiastic, discourse, supposedly
upon the subject of the title.

Neither Gardner\supercite{gardner:vision-and-vesture}
nor Selincourt\supercite{de-selincourt:william-blake}
reveal what could be considered a substantial grasp of Blake's \enquote{meaning.}

Edwin Ellis, in \emph{The Real Blake},\supercite{ellis:real-blake} expatiates energetically, as
do those mentioned above, mainly in verbose inanities,\footnote{Ellis' rather active mind produced such samples as the following, which elaborates upon the idea of mental activity, or \enquote{this is Christendom}: \enquote{Blake\dots coined a term for it (\enquote*{virtual velocity}) himself. It was mind in a state of patience. He accounted for motion by the idea that wind is eternal, but patience is not. Patience removed, mind moves\dots}}
yet certain of his statements are not entirely without value, for instance: \enquote{\dots to attain\dots inspiration is
the duty of all who can do so\dots} and \enquote{In Heaven all is brotherhood. \enquote*{In eternity all is vision.} There is
a socialism of the soul there, and communism of property.} Supporting analysis, however, is lacking.

The introduction of the Chatto and Windus edition of
\emph{The Marriage of Heaven and Hell}\supercite{blake:heaven-and-hell}
contains the somewhat interesting statement: \enquote{\dots Blake believed that he knew the
secret, and possessed the faculty of obtaining more than a glimpse of the pure light; he believed, too, that
what men call reality is in truth but a dreamland, and that imagination alone \enquote*{in this age} can lead us to the real,} and
the supporting statements that Blake's \enquote{faculty} was \enquote{an extreme development of the
faulty of creative imagination,} and that this \enquote{hypertrophy} \enquote{approaches, in short, a new sense,} which is
\enquote{incomprehensible to the ordinary man.} Despite this apparent analysis, it seems that the above work must
be included with this first group because it makes use of, to support the explanation, the
term \enquote{creative imagination,} which does not have a \enquote{standard} definition (which would
place it in the second category), without giving a definition for it (which would place it in the
third category). Another factor that seems to place this work entirely within
this category is the extremely naive discussion of Blake's supposed \enquote{symbolic Christianity}: without
any substantiation from Blake's writings (which, incidentally, are directly contradictory: see chapter three of this thesis),
it is indicated that the \enquote{central tenets} of his religion are in some way concerned with the interrelations
of \enquote{inspiration} and \enquote{symbols}; although this near-deification of symbols is typical of Freudianism, which
was developing around this time, the only Freudian-religious study of Blake to be considered in this chapter is more
appropriately included in the second category, since the naive attitude toward symbols is hidden by its concern with
the \enquote{cause} of religion, rather than its (supposed) \enquote{goal.}

\section{Second Category}

Julien Green, in \enquote{William Blake, Prophet,}\supercite{green:blake-prophet}
says that Blake was a \enquote{true mystic,} in that he \enquote{separated} the human,
the mere appearances, from the spirit, the eternal aspect, of man:
that is, he was a dualist; however, it seems that the only support given for this belief is some
biographical information which reveals that Blake experienced the extremes of love and hate. Because of the particularization given to the term \enquote{mystic,} and the failure to show a relationship,
even in the significant particulars, and much less in [illegible], this study is excluded from both the first
and third categories, and is therefore to be considered (merely to indicate that it deserves, unlike Butterworth's and Ellis' of
the first, some respect) a member of the second. Approximately the same can be said about Mark Schorer's \enquote{William Blake and
the Cosmic Nadir,}\supercite{schorer:cosmic-nadir}
except that in this article the author's belief that the concept of the \enquote{Fall} (of Man) lies
at the bottom of Blake's whole system is emphasized; this concept seems to be of such integral importance to the following
scholar, Evelyn Underhill, that it requires no particular elaboration.

\clearpage

\phantomsection
\label{self:07}

On the basis of the discussion of Underhill's \enquote{definition} of
mysticism in the first chapter of this thesis, it should
be obvious that her treatment of Blake will not be included in the
third \enquote{category,} since Blake's mysticism constitutes only a small part of her large study, and
seemingly could easily be lightly, and thus inaccurately, treated; however, Underhill's study of
mysticism was sufficiently broad, and her understanding of Blake accurate enough that she, without
warping Blake extremely,\footnote{See pp. \pageref{self:12} of this thesis; also the comment on pp. \pageref{self:13}}
explained his position with some degree of thoroughness, though
the accuracy of the basic explanation can be doubted. It has already been
pointed out\footnotemark[\value{footnote}]
that Underhill, as representative of the Catholic
scholars of mysticism, considers Blake, as well as Boehme, to be a mystic
who is \enquote{imperfectly dualistic}; he is further described as representing a \enquote{stage of growth
which the mystics calls the illuminative way,} which is only the \enquote{first mystic life.}\supercite{underhill:mysticism}
This sort of mystic, probably because of his natural temperament (\enquote{In the artist, the senses have somewhat
hindered the perfect inebriation of the soul}\supercite{underhill:mysticism}), has come to rest, Underhill believes, after an
initial period of \enquote{oscillation} between opposing and embracing the \enquote{intuitions} of the Absolute, during which
period certain \enquote{discordant} characteristics have been eliminated from the personality, and is convinced that he has
attained the \enquote{divine and veritable world} that his understanding is \enquote{final and complete.} he is, however, Underhill
says, \enquote{ignorant as yet of that consummation of love which overpasses the proceedings of the inward eye and
ear\dots and, absorbed in this new blissful act of vision, forgets that it belongs to those who
are still \emph{in via}.}\supercite{underhill:mysticism} Most of the people who reach this stage, apparently
including Blake, fail to go on to the \enquote{Source,}
the \enquote{true goal} which lies far beyond this merely \enquote{illuminated} state.

The preceding analysis of Blake, and similar mystics, by
Underhill is based on a distinction between two types of
\enquote{passive imaginary vision}--by which is meant \enquote{spontaneous mental
pictures at which the self looks, but in the action of which it does not participate---,}\footnote{Blake seems to represent a variation of the above described \enquote{imaginary vision,} since, although there was apparently no sensorial hallucinations, i.e., the self saw sharply and clearly, but \enquote{perfectly aware that it\dots did so in virtue of its most precious organ}---the faculty of imagination, Blake at least claimed to have participated in certain of his visions, as when he shocked aqquatinces by telling them he had just returned from conversation with a fairy, or that, he had touched the sky; if Underhill were familiar with these statements it would probably add to her opinion that many of the experiences of \enquote{such visionaries as Boehme and Blake\dots are far more occult than mystical in character.}\supercite{underhill:mysticism}}
viz., symbolic and personal. Blake is apparently considered to be the \enquote{artistic type} who
experiences \enquote{symbolic vision,} which are the form taken by the
meditations of those \enquote{good visualizers} such as St. Ignatius Loyola (and the students of Raja Yoga),
who advise that a \enquote{deliberate effort} be made to visualize the subject dwelt upon, rather than to attend
to it \enquote{discursively.}\supercite{underhill:mysticism}
The personal form of the \enquote{vision} is considered to be, rather than
that is apparently merely an unusually efficacious form of meditiation, \enquote{a true contact of the soul with
the absolute life\dots,} one of \enquote{\dots those genuine mystic states in which the immanent God, formless, but capable
of assuming all forms, expressed himself in vision\dots}\supercite{underhill:mysticism}

Another form of \enquote{vision} which apparently \enquote{ranks} above the
two types of \enquote{imaginary vision,} namely, \enquote{intellectual vision,} in
which \enquote{Nothing is seen, even with the eyes of the mind\dots}\supercite{underhill:mysticism}
is also, though inconsistently and falsely, attributed by Underhill to Blake: \enquote{\dots all pure mystics\dots have drunk, with Blake, from
that cup of intellectual vision which is the chalice of the spirit of life\dots}\supercite{underhill:mysticism}
This, however, seems to be simply a minor confusion, since nowhere in her many references to Blake does she consider him to be more than merely one of the most
intensely mystical artists. Although in her summary of Blake, she calls him a \enquote{mystical genius}, Underhill undoubtedly believes that
he, as the other \enquote{English Protestant mystics,}\supercite{underhill:mysticism}
was \enquote{an abnormality} and that he failed to exhibit \enquote{in its richness the unitive life.}\supercite{underhill:mysticism}\footnote{See the first chapter of this thesis.}

If Underhill seems to have understood Blake fairly well, and \enquote{disapproved} of him somewhat, in accordance with her
dualistic position, and other dualists discussed earlier seem to have misunderstood him while \enquote{approving} of him, the
following writer, Sheldon Cheney, compromises between these two positions.

Cheney, a unique combination of a Platonic mind, enthusiasm for Blake, and a high degree of consistency in his explanation
of Blake's beliefs, seems to be one of the two most significant Christian commentators on Blake's ideas; the other,
Evelyn Underhill, is the more valuable for a clear definition of Christian mysticism in terms of which Blake is evaluated;
but Cheney seems to reveal the more perfect understanding of Blake: Cheney's \enquote{system} tended
to adapt to Blake, while Underhill seems to have \enquote{adapted} Blake somewhat to Platonic dualism,
for example, Cheney shows his Platonic attitude, without altering Blake too much, when he said that
Blake \enquote{believed that the soul is, during its time-conditioned life on Earth, a wanderer from
the realm of pure spirit, from an Eden that exists eternally\dots,}\supercite{cheney:walked-with-god}
yet when he, in common with Colin Wilson, says, \enquote{he spoke too of the pleasures of sex as a gateway to vision, a holy gateway
opening upon the clearest eternal seeing of which the soul is capable,}\supercite{cheney:walked-with-god}
he seemed to have discarded his Platonism temporarily, since he makes no attempt to adapt that statement to a larger theory; a
writer of Evelyn Underhill's type would have either omitted this reference or would have shown it to be an indication
either of Blake's inconsistency, or of his \enquote{materialism,} the preceding, however, is not a complete indication
of Cheney's application of Platonism to Blake's beliefs: near the end of his chapter on Blake, there is at least one
instance in which he seems to contradict the statement quoted above, namely, \enquote{nothing could be farther from the sensual, wilful Superman
of Nietzsche,}\supercite{cheney:walked-with-god}
referring to \enquote{Man} in Blake's line \enquote{Thou art a Man, God is no more.} This statement, which
contrasts Blake's \enquote{man} to Nietzsche's \enquote{Superman,} is, however, apparently contradicted by nearly two pages of discussion
of the similarities of the differences between Blake's and Nietzsche's ideas of the \enquote{Superman,} this discussion
begins with the statement, \enquote{William Blake was, as certain philosophers have pointed out, a forerunner of Nietzsche's and
of the cult of the exalted individualism.}\supercite{cheney:walked-with-god}
\enquote{Blake,} Cheney says here, \enquote{\dots can be said to go along with them (the Nietzscheans)
brilliantly in his preliminary destructive phase\dots}
The difference between Blake's and Nietzsche's ideas of the\supercite{cheney:walked-with-god}
\enquote{Superman,} Cheney says, is that Blake adds spirituality to the idea of
\enquote{an individual who is a Superman only physically and intellectually.}\supercite{cheney:walked-with-god}

\phantomsection
\label{self:19}

Cheney divides Blake's mystical \enquote{say} into two parts, as follows: \enquote{The only
half of the struggle upward is the cultivation of multifold vision, the training of the faculties
of spiritual seeing, of clairvoyance. There then remains the more purifying, the more holy, half of the
\enquote*{way,} the process that ends with the restoration of oneness within divinity.}\supercite{cheney:walked-with-god}
Since he says that the first part is \enquote{perceiving the divine in everything} is, as he says in this same paragraph, perceiving
that (in Blake's words), \enquote{everything on Earth\dots in its essence is God,} this perception is enough to
realize the \enquote{oneness with divinity,} unless, of course, it is insisted, in the Platonic or dualistic fashion,

\clearpage

\noindent that \enquote{a thing is separate from its essence,}\footnote{It will be noted in chapter three that \enquote{essence} is shown to be seperate from \enquote{things} i.e., from \enquote{everything on Earth,} and that it is used by Blake as an equivalent to \enquote{God} the \enquote{omnipotent, uncreate} (as distinguished from \enquote{infinite} man as \enquote{God}), but the problem which Cheney makes for himself (and fails to solve), by using the characteristically Christian mystical idea of a two (or more) step \enquote{way} and the other Christian and/or Neo-Platonic idea of \enquote{union with (the universal) God,} is avoided, since, to the best of my knowledge, Blake made no comment other than those which indicate that a recognition of the universal God (called by Underhill \enquote{entelechy} and the spirit of \enquote{becoming}) leads to a recognition of man's \enquote{particular,} yet \enquote{infinite} (see pp. \pageref{self:14} of this thesis) and god-like \enquote{identity.}}
and if this is supposed to be the case Cheney neglects to give any explanation of how the gap is to be bridged.

Cheney's general evaluation of Blake as a mystic is wholehearted:\linebreak
\enquote{William Blake \emph{was} the mystic, imagining, visioning,
walking with God.} \enquote{He is the most genuine and most illuminating mystic in the British Line, and not to be
matched in any country in the Western world during his century.}\supercite{cheney:walked-with-god}
\enquote{Blake's unique importance in the history of
prophecy and mysticism is that he expressed with more beauty than any other, in poetry and in paintings, this message of the spirit
(\enquote*{The individual\dots can be lifted up\dots in mortal life, to that realm of divine illumination and experience}), in terms essentially
Christian yet universal.}\supercite{cheney:walked-with-god}

\phantomsection
\label{self:08}

The editors of \emph{The Cambridge History of English Literature}\supercite{cambridge:english-literature}
seem to have had approximately the same attitude toward
Blake as does Cheney, that is, they seem to have been aware of a relation between a concern with ethics and
slightly dualistic mysticism: \enquote{His mystical faith freed him from the barren materialism of his age and opened to him in
vision the world lying beyond the range of physical senses. Hence, the greater warmth of his ethical creed; and his
preoccupation with the supernatural, which he never consciously shaped to literary ends, is yet the source of the peculiar
imaginative quality of his work\dots} Although these peculiar comments are more meaningful than most of the others written
during the same period (before 1915), some of which are considered on the first category of this chapter, they are
typical of that group in their neglect to define \enquote{mystical faith,} and to give support for the statement that Blake was
preoccupied with the \enquote{supernatural.}

Schorer's article, \enquote{Blake as a Religious Poet,} in the \emph{Sewanee Review},\supercite{schorer:blake-poet}
seems to belong to the second category, since a \enquote{standard} definition of mysticism, viz., \enquote{\dots mysticism is the systematic search for
the transcendental absolute, the uncreated God, through the repudiation of worldly claims and
social values\dots}\supercite{schorer:blake-poet}
is used, although it is used only negatively: according to this definition, Schorer
believes that Blake was not in the least \enquote{mystical,} and was not even a \enquote{religious} poet. Except for the way that Schorer chooses
in this article to use the term \enquote{visionary,} rather than a redefined \enquote{mystic,} to distinguish between the two types of
\enquote{mysticism,} this article might be included in the third group, which includes those studies which consider Blake objectively
without disregarding his \enquote{psychological} aspects; the use of \enquote{visionary} here seems to exclude all but the political or
practical aspects of the \enquote{positive mysticisms} outlined in the first chapter. Schorer's main error seems to be a failure
to see a common denominator for the supposedly mutually exclusive areas of society and \enquote{mystical theology.}

\section{Third Category}

Finally, for the second category, a brief statement by the novelist, D. H. Lawrence, is worth
considering, although its only justification for inclusion in this category is its \enquote{negative} attitude toward what is
implied to be a purely negative \enquote{mysticism,} i.e., it rejects the application of either the popular meaning \enquote{mysterious} or the
scholarly meaning \enquote{divorced from life} to \enquote{Blake's} and Lawrence's work, rather
than attempting to elucidate positively Blake's position. It seems to be Lawrence's implication that certain
forces in society habitually attempt to vitiate works of a certain type (by asserting that their purpose is \enquote{other worldly}), apparently
those which are \enquote{revolutionary} attempts to make men more intensely aware of their existence. This statement was made concerning
\emph{The First Lady Chatterley}: \enquote{They'll say as they said of Blake: It's mysticism, but they shan't get away with it, not this time: Blake's
wasn't mysticism, neither is this.}\supercite{lawrence:first-lady-chatterly}

Alfred Kazin,\supercite{kazin:portable-blake}
although his idea of Blake is probably very much like H. H. Lawrence's, seems to be more accurately
classed in the third category, since, besides saying that Blake is not, \enquote{in any ordinary sense, a mystic,}\supercite{kazin:portable-blake}
he shows that he resembles the \enquote{Christian mystic} (dualistic) in certain ways, (e.g., a sense of doubleness--but this is denied
by Kazin in his saying that Blake didn't admit a distinction between the \enquote{real} and \enquote{ideal}; also, that Blake was
involved in a mystic quest), and is labeled as a \enquote{visionary,} rather than a \enquote{mystic,} simply to avoid confusion and also
(is to be classified in the third category) because of his factual treatment of Blake's \enquote{philosophy,} although many inconsistent
interpretations can be found. It seems to be Kazin's failure to find basic philosophic principles on which to interpret Blake's works that
result in his contradictory statements. Typical of his inconsistency are his statements that Blake was against society\supercite{kazin:portable-blake}
and that he was not the enemy of society.\supercite{kazin:portable-blake}
Despite the general accuracy of Kazin's work, his inconsistencies frequently take the form of a
distortion of Blake's doctrine into some form of Dualism, which is typical of those who define \enquote{mysticism} dualistically;
even though he is said to be definitely different from the Christian (dualistic) mystics, the lack of a basic theory
allows the idea of \enquote{mystic} indirectly applied to Blake, to carry with it some of the dualism given it in its \enquote{Christian}
definition.\footnote{See pp. \pageref{self:15} of this thesis.}
The value in Kazin's analysis lies in his frequent insights into Blake's world view, such as, \enquote{Blake
assumed that what is partial is in error, and that what is limited is non-existent,}\supercite{kazin:portable-blake}
and that he refused\dots\enquote{to concede a distance between what is real and what is ideal\dots}\supercite{kazin:portable-blake}
Although the latter is not entirely applicable; his errors are based
on the belief that \enquote{\dots the truth is that he was not trying to prove anything philosophically at all\dots}\supercite{kazin:portable-blake}
These errors are exemplified by statements such as \enquote{Blake was the mystic's tormented sense of the doubleness of life between reality and the ideal,}\supercite{kazin:portable-blake}
and that \enquote{\dots the doubleness of all existence\dots} is \enquote{\dots the unalterable condition of the human struggle.}\supercite{kazin:portable-blake}

\phantomsection
\label{self:13}

Blake, Kazin says, was \enquote{supremely intelligent,} and \enquote{\dots had one of the greatest minds in the history of our culture};\supercite{kazin:portable-blake}
rather than being \enquote{off the main track,} as \enquote{the textbooks} claim (\enquote{\dots and that shuts hum off from us}), he is \enquote{simply
ahead of it.}\supercite{kazin:portable-blake}
He implies that Blake's intelligence is the cause of the fact that \enquote{most of his biographers have had no
understanding of him}; Alexander Gilchrist and Mona Wilson, exceptional writers, \enquote{\dots at least sought the basic facts about him.}\supercite{kazin:portable-blake}
\enquote{The usual view,} he says, \enquote{is that he was a happy mystic, who sat like a gloriously content martyr before his work, eating bread and locusts with an idiotic smile
on his face. Blake evidently did enjoy great happiness in many periods, for he was a man for whom file consisted in exploring his own
gifts. But there is even more in Blake's total revelation of himself, a rage against society\dots}\supercite{kazin:portable-blake}
The object of Blake's rage, \enquote{modern capitalist
society,} was \enquote{\dots a world of\dots brutal exploitation and\dots inhuman ugliness,} in which Blake, as well
as \enquote{millions} of other Englishmen, \enquote{\dots felt himself being slowly ground to death\dots}\supercite{kazin:portable-blake}
Kazin's apparent self-contradiction concerning
\enquote{society} (mentioned above) will be resolved if an appropriate use is made of the interpretation
of \enquote{society} as \enquote{parts of society.} On this basis, Kazin's discussion of \enquote{society} is somewhat reconciled
to his statements concerning Blake's \enquote{mysticism,} or lack of it, which seem to suggest a sort of \enquote{social
mysticism}: \enquote{\dots Blake was not looking for God. He shared in the mystic's quest, but he was not going the same way,}\supercite{kazin:portable-blake}
and \enquote{\dots(Blake was) a mystic who reversed the mystical pattern, for he sought man as the end of his search.}\supercite{kazin:portable-blake}

Although Kazin denies that Blake was \enquote{trying to prove
anything philosophically at all,} he reveals a distinct
philosophical position, whether it is intentional or not;
and his use of paradox (rather, self-contradiction), as mentioned
in the preceding paragraph, suggests that it is unintentional. This
philosophical position is more clearly explained in the following discussions
of studies of Blake, especially in the last, in this third category, and in the
third chapter of this thesis. An idea which is treated paradoxically by Kazin, and which
will be elaborated upon in the third chapter, is that concerning the relation of
God to man, Kazin, in the sentence quoted above (\enquote{\dots he
sought man\dots}), also says, \enquote{he was a libertarian obsessed with God};
more detailed analysis will reveal that the seeking of \enquote{God} is not necessarily contradictory
to the \enquote{seeking} of the \enquote{true} man and society.

Norman Nathan, in a very intelligently written dissertation,\supercite{prince:conception-of-blake}
says that Blake's philosophy is a modern one, that it is even \enquote{\dots far into the future,} and that it is
\enquote{\dots also close to the needs of the average man,}\supercite{prince:conception-of-blake}
thus allying himself with such writers as D. H. Lawrence and Alfred Kazin, who maintain that his Philosophy is ethically
truthful and practical. Nathan's most important contribution seems to be his explanation of
\enquote{imagination} as both the creator of forms or \enquote{entities,} and the \enquote{coordinator} of them,
which distinguishes relations between the forms,\supercite{prince:conception-of-blake}
but his prior explanation of the nature of entities as \enquote{perceptions,} which are \enquote{momentary causes,}\supercite{keynes:william-blake}\footnote{See pp. \pageref{self:16} of this thesis.}
is also important. Parallel ideas will be discussed in the first and second sections of the third chapter.

It was mentioned earlier in this thesis that Colin
Wilson resembled both Cheney and Leuba in certain ways; it
will so be seen that most of the writers being considered
in this category present interpretations of Blake which are
allied with each other, or to the personal attitudes of
those (discussed in the first chapter) who are \enquote{positive}--or
\enquote{sensuous}--mystics. Wilson's general theory as presented in \emph{The Outsider},\supercite{wilson:the-outsider}
is that the \enquote{correct} desire of \enquote{life,} intelligent human beings in particular, is \enquote{more life};
in his opinion, Blake represents one of the most
prefect illustrations of the truth of this theory. It seems to be Wilson's
belief that the expression of this desire in one field of human activity will lead to a greater expression of that desire in the same field, or
in others. Specifically, Wilson says, \enquote{Blake had preached that sex can raise man to visionary insight,}\supercite{wilson:the-outsider}
The writers considered next present what is in general the same interpretation, considered
from slightly different points of view.

Harold C. Goddard, in an interesting pamphlet,\supercite{goddard:fourfold-vision}
presents an interpretation of Blake's \enquote{mysticism} that, although it is not as analytical as the first two studies considered
in this category, seems to contain a philosophically \enquote{central} concept,
that is, it succeeds, to some extent, in \enquote{explaining} the \enquote{mysticism,} without departing from
the known facts. Goddard discussed Blake's \enquote{ethics,} that is, his politics
and theories of morality, and suggests that his social attitudes and practices
are related to, or responsible for, his \enquote{mysticism.} Stating that happiness is not incompatible
with mystically intense consciousness, as did Blake, Goddard says that when our \enquote{\dots ship
sails the seas of reality successfully, we have fourfold vision.}\supercite{goddard:fourfold-vision}
This statement implies, most simply, that the act of exercising control over one's environment (sailing successfully) leads
to more complete perception; more specifically, however, it implies that
\enquote{correct perception} results in \enquote{fourfold vision,} or vision (or imagination)
which is not \enquote{bound by virtue,} that is, not limited by abstraction. Goddard's
interpretation of Blake, thus, is seen to resemble the mystical philosophy of
the Mahayana Buddhists,\footnote{See pp. \pageref{self:17}-\pageref{self:18} of this thesis.}
and, incidentally, the psychological theory of Albert Einstein.\footnote{See pp. \pageref{self:19} of this thesis.}

In a study of Swedenborg and Blake,\supercite{estenson:swedenborg-and-blake}
L. O. Estenson says that one of Blake's earlier books,
\emph{The Book of Thel}, \enquote{\dots is concerned primarily with the Swedenborgian idea of attainment of
spiritual unity through experience (through sensual and corporeal expression and impression).}\supercite{estenson:swedenborg-and-blake}
\enquote{Thel, as the unborn, must undergo mortal generation, the dominance of the five senses, and the resultant
subjugation of spiritual man before attaining perfection in the unity of opposites.}\supercite{estenson:swedenborg-and-blake}
The \enquote{subjugation of the spiritual man} seems to refer to, especially when the other
writers of this category are considered, the rejection of intellectual, or abstract modes of
thought, in favor of a wholehearted entering of the world of action, for the purpose
of gaining security in that world. Beginning with the basic necessities, and advancing as far
as possible, which, in the opinion of the psychologist Maslow, who maintains the same opinion (regarding the
\enquote{union of opposites}), is mystical consciousness.

Finally, a writer will be considered who fearlessly abandoned any consciousness of a necessity
that \enquote{mysticism} be defined supernaturally. Benedict Alper says that
\enquote{Mysticism was the key\dots to all the aspects of his (Blake's) life.}\supercite{alper:blake-mysticism}
It is not, however, supernatural mysticism.\supercite{alper:blake-mysticism}
As Kazin only vaguely suggested, \enquote{God} is, Alper believes, not separate from man; Blake got his
inspiration from the \enquote{body of God,} but this is simply \enquote{imagination.}\supercite{alper:blake-mysticism}\footnote{See the discussion of the \enquote{entelechy} in the third chapter of this thesis for a more complete explanation of the meaning of \enquote{imagination,} or (Blake's) God.}
It seems to be Alper's opinion that \enquote{divinity} is in some way obtained by the \enquote{strong and
unrestrained} expression of the desires. This \enquote{apotheosis} is the subject especially of the
second part of the following chapter, although the third part discusses the \enquote{means}
to this goal, and their application, as conceived and executed by William Blake.