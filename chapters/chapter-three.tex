\chapter{Analysis of Blake's Mysticism}

\section{Blake's Epistemology, Including Experience and Theory of Experience}

If Blake were like Christian mystic, the only problem to be considered in analyzing his
writings would be whether or not he indicated that he had had a certain unusual experience,
usually called \enquote{union with God,} but taking almost as many forms as there have been mystics; Blake, however,
was sufficiently unlike these mystics that this is only part of the problem. Blake's basic difference was that
his \enquote{mystic may} seems to have led into, rather than away from, the \enquote{world.} This assumption
suggests three main questions which can serve as points of consideration to verify
or modify it, namely: \enquote{what did Blake experience that is not generally experienced by others in the world?}
and \enquote{how does Blake explain any difference between his and the normal experiences, or what does he say about the
nature of the world and of man that could elucidate his experience?} and \enquote{his being a world-centered mysticism,
does Blake consider a worldly ethic and eschatology to be immediately (i.e., having a cause or effect relation) involved
with it?} In other words, the three area to be investigated in attempting to obtain an understanding of
Blake's \enquote{mysticism} could be called \enquote{the epistemology, metaphysics, and world-affirming
mysticism,} because of the peculiarity of Blake's \enquote{system,} the usually distinct
areas of epistemology and metaphysics are almost inseparable, that is, each is, to some degree,
implicit in the other. If Blake had written in the language of traditional philosophy
he probably would have insisted that the three areas must be united; as it is, his unnamed and somewhat
homogeneous exposition of his \enquote{mysticism} must be arbitrarily divided to show it in a generally comprehensible form.

There are one or two poems in Blake's letters to his friend Thomas Butts which are clearly narrations of particular
experiences which Blake considered to be significant. Besides these, there are important references to \enquote{visions} scattered
through \emph{Milton}. As a basis for the subsequent consideration of these descriptions of \enquote{visions,} there is
a marginal statement in Blake's copy of Berkeley's \emph{Siris} which, although the wording
is not complete enough to allow it to be considered an epitome of Blake's \enquote{epistemology,} at least
indicates its general direction. In response to Berkeley's statement which ends with the sentences
\enquote{Reason considers and judges of the imagination. And these acts of reason become new objects to the understanding,} Blake says:

\quotebox{%
	\enquote{Knowledge is not by deduction, but Immediate by Perception
	or Sense at once. Christ addresses himself to the Man, not to his reason. Plato did not bring Life
	\& Immortality to Light. Jesus only did this.}\supercite{keynes:william-blake}
}

According to this statement, \enquote{absolute knowledge,} i.e., knowledge about
\enquote{life and immortality,} is to be gained by means of \enquote{perception or sense,} which clearly
indicates that Blake's position is directly opposed to the general position of those
Christian and Hindu mystics who, like Berkeley, believe that \enquote{The perceptions of sense are gross\dots}\supercite{keynes:william-blake}
the above quotation from Blake does not, of course, \enquote{\dots bring life and immortality to light,} nor does it specifically
reveal how they were brought to light for Blake, that is, it reveals neither
what is sensed (which will be discussed in the section concerning his \enquote{metaphysics}), nor
what he means by \enquote{sense} (except that it is not \enquote{deduction} or \enquote{reason}); for Blake's specific meaning of
\enquote{sense} the references concerning \enquote{visions} and \enquote{mystical} experiences must be considered.

The poem sent on October 2, 1800, to Blake's \enquote{friend Butts} reveals an experience which has very much in common with
the experiences described by the modern \enquote{psychological mystics,} Aldous Huxley and A. H. Maslow (see chapter one).
In the first twelve lines of the poem Blake indicates that his \enquote{first vision of light} occurred when he was sitting
on the beach, and that something happened to him which enabled him to experience that increase of \enquote{vision,} and that
it was some sort of affirmation of sense:

\clearpage

\phantomsection
\label{self:26}

\poembox{%
	Over sea, over land  \\
	My eyes did expand   \\
	into regions of air  \\
	away from all care,  \\
	into regions of fire \\
	remote from desire\dots\supercite{keynes:william-blake}
}

The last line quoted suggests that rather than desiring \enquote{absorption in God,} and thus \enquote{freedom from the world,}
as many supernaturalist mystics have done, Blake has more perfectly entered the world of individual moments and events
by accepting, rather than rejecting (to some degree), the facts which were present to him; for instance,
he earlier in this poem (line five) called the sun's light its \enquote{glorious beams.} That
this statement of remoteness from desire may be more than merely an indicator of positive interest in
what is present, and thus need not be desired, is suggested by another verse, which ends:

\phantomsection
\label{self:23}

\poembox[2.25em]{%
	\dots Desire gratified \\
	Plants fruits of life and beauty there.\supercite{keynes:william-blake}
}

Although the relationship between \enquote{life} and the \enquote{visionary} or \enquote{mystical} experience will be developed in the
section treating of Blake's \enquote{metaphysics,} it is not difficult to see that the second, more physical, reference
to absence of desire is at least parallel to the first reference, that is, if the first does not intend to imply
that \enquote{desire gratified} leads directly to \enquote{mystical consciousness,} as is believed in Tantra doctrine, it
is at least parallel to the second in its implication that \enquote{desirelessness,} whatever its cause,
is associated with intensity of life or sense-consciousness. In actuality, as it will be shown later, \enquote{desirelessness}
is the equivalent of \enquote{possession,} that is, gratification of the desire to possess, and thus necessarily includes the
idea of \enquote{desire.}

Whereas the \enquote{supernaturalist} tends to \enquote{generalize} the details of the world, by
including them within the outlines of a symbol (e.g., \enquote{spirit is light,} a supernaturalist metaphor,
contains a fairly generalized symbol, the word \enquote{light,} and intends that the multiple
referents of that symbol be considered as one fact, which in term is to be used as a
symbol of something \enquote{higher}), Blake tends to particularize; for instance, in lines fifteen to seventeen he says:

\poembox{%
	In particles bright, \\
	The jewels of light  \\
	Distinct shone and clear.
}

It is apparently this same sort of intense perception that caused Huxley to speak of
\enquote{\dots a bundle of minute, unique particulars in which, was to be seen the divine source
of all existence.}\supercite{huxley:doors-of-perception}
The particularization is increased in the next lines,
which seem intended not to be taken literally:

\poembox{%
	Amaz'd \& in fear      \\
	I each particle gazed, \\
	Astonish'd, amazed;    \\
	For each was a Man     \\
	Human-form'd.
}

An extension of this anthropomorphism is obtained by having the \enquote{particles}
speak the lines:

\poembox{%
	\dots\enquote{Each grain of sand,                       \\
	Every stone on the land,                        \\
	\dots\dots\dots \\
	\dots Cloud, meteor, \& star                    \\
	Are men seen afar.}
}

Besides increasing the particularly, the reality and the significance of the \enquote{particles of light}
by calling them \enquote{men,} it seems to describe their nature, or their functions (actions):
they seem to be alive,\footnote{See section II of this thesis.}
creative, and, as his eyes were, \enquote{expanding}; they even
\enquote{beckon'd to} him; in other words, during his intense perception of physical reality, he dropped
his habitual perception of non-human matter as being essentially foreign, or even evil. It is this uninspired
perception of reality that Blake calls \enquote{natural religion} and attacks frequently.
Although it really belongs under a later section it seems important further to verify this
assertation with a more explicit statement by Blake before showing his forms of \enquote{experience} since it
is so widely believed that he is some sort of \enquote{Platonic mystic} (Bert. Jessup) (Despite his many blunt
criticisms of Plato). Apparently on the basis of titles such as \enquote{There is no
natural religion,} sufficient verification for the present can be found in the section of \emph{Jerusalem}
\enquote{To the Christians} in which he speaks of the \enquote{Wheel of religion,} whose:

\phantomsection
\label{self:20}

\poembox[2.25em]{%
	\dots Name Is Caiaphas, the dark Preacher of Death, \\
	Of sin, of sorrow, \& of punishment;                \\
	Opposing Nature! It is natural religion             \\
	But Jesus is the bright Preacher of Life            \\
	Creating Nature\dots\supercite{kazin:portable-blake}
}

This digression has helped to prepare the way for an explanation of the latter
part of the poem \enquote{To Butts}; after several lines during which Blake looks back on the
recent past from his inspired point of view, he returns his attention to his \enquote{experience}:

\poembox[2.25em]{%
	My eyes, more and more       \\
	Like a sea without shore,    \\
	Continue expanding,          \\
	The Heavens commanding;      \\
	Till the jewels of light,    \\
	Heavenly men beaming bright, \\
	Appear'd as One Man\dots
}

The idea of this seems to be that of the realization that since all \enquote{units} of nature
are manlike, the totality of nature is also man-like. The natural consequence of this
realization, i.e., seeing that one's own humanity is in no way separated from the totality
of nature, is described in the actions of that \enquote{one man,} who:

\poembox[2.25em]{%
	\dots Complacent Began       \\
	My limbs to infold           \\
	In his beams of bright gold; \\
	Like dross purged away       \\
	All my mire \& decay.
}

In other words, the \enquote{completion} of the experience removed the last remnants of Blake's
\enquote{natural religion,} that is, of his opposition to nature. He then describes this condition of
pure perception of nature:

\poembox[2.25em]{%
	Soft consum'd in delight     \\
	In his bosom sun bright      \\
	I remained. Soft he smil'd   \\
	And I heard his voice Mild   \\
	Saying: \enquote{This is My Fold,    \\
	O thou Ram horn'd with gold, \\
	Who awakest from sleep\dots}
}

\phantomsection
\label{self:21}

What Blake means by sleep here is the same thing that \enquote{supernaturalist} mystics consider to be their
\enquote{enlightenment,} i.e., removal from contact with the world; Blake believes that the \enquote{Nirvana,} or
\enquote{snuffing out,} of certain Hindu sects, and the \enquote{ineffable vision} of certain Christian
mystics is a \enquote{spiritual sleep,} or a failure to perceive the true nature of the real (i.e., physical) world.
Speaking of this difference in experience (or claimed experience), Blake says, in \emph{Jerusalem}:

\phantomsection
\label{self:11}

\poembox[2.25em]{%
	He who would see the Divinity\dots,                                 \\
	\dots\dots\dots \\
	\dots he who wishes to see a Vision, a perfect Whole,               \\
	Must see it in its Minute Particulars, Organized,                   \\
	\dots \& not as\dots a Disorganized                                 \\
	and snowy cloud: brooder of tempests \& destructive War.\supercite{kazin:portable-blake}
}

\noindent After the \enquote{voice mild} of the \enquote{one man} faced Blake writes:

\poembox[2.25em]{%
	I remain'd as a child;  \\
	All I ever had known.   \\
	Before me bright shown: \\
}

This refers to the state of \enquote{innocence,}\footnote{See chapter III, third section, \enquote{ethics.}}
which may be considered as the natural condition, which has not been lead away from its contact
with reality by the \enquote{wheel of religion,} or that \enquote{natural religion} which
opposes life and nature. In the third section of this chapter the inter-relations
of \enquote{innocence,} \enquote{vision,} and \enquote{mysticism,} will be discussed; but it is relevant to
this section to mention that Blake's experience, unlike many of the Christian and Classical Hindu
mystics, although similar to the \enquote{liberal} Buddhists and Hindus, is, without external
interferences, a permanent condition. While certain \enquote{supernaturalist} mystics have practiced
self-mortification and withdrawal from society as means to achieve their experience as frequently
as possible, if not continuously, Blake practiced self-gratification (\enquote{Abstinence sows sand all over\dots})
and entered society (for example, his association with Paine and Godwin) with the intention
of aiding those forces which furthered his \enquote{vision} and destroying those which hindered his \enquote{vision,}
as means to achieve the most permanent, as well as the most intense, experience.

The first book of Milton contains a description of the internal \enquote{events} of the mystic experience,
described in a somewhat figurative, though clearly naturalistic, way: \enquote{The sons of Ozoth\dots} (who are \enquote{Sons of Los,}
the \enquote{eternal prophet}).

\poembox[2.25em]{%
	\dots within the Optic Nerve stand fiery glowing.                     \\
	And the number of his sons is eight millions \& eight.                \\
	They give delights to the man unknown, artificial riches              \\
	They give to scorn, \& their possessors to trouble \& sorrow \& care, \\
	Shutting the sun, \& moon, \& stars, \& trees, \& clouds, \& waters,  \\
	And hills out from the Optic Nerve, \& hardening it into a bone       \\
	Opake, and like the black pebble on the enraged beach\dots
}

Although the naturalism of \enquote{optic nerve} should be obvious, it is likely that \enquote{artificial riches}
will be interpreted by some to mean material wealth, with the result that Blake
will again be placed, in those minds, with those who \enquote{reject} the world.
That Blake's \enquote{experience} did not have its basis in self-denial is revealed in such statements as:

\poembox[2.25em]{%
	\dots reflect on the State of Nations under Poverty \& \\
	their incapability of Art; tho' Art is Above Either,   \\
	the Argument is better for Affluence than Poverty;     \\
	Happinesses have wings and wheels; miseries are leaden legged\dots\supercite{kazin:portable-blake}
}

Since \enquote{artificial riches} seems not to refer to the simple possession of material wealth, it might be interpreted
to mean either material articles which are possessed with the wrong intention, or intellectual
possessions, i.e., systems, methods (doubt), or moral beliefs which are possessed for themselves,
rather than as tools: the correct attitude toward \enquote{systems} is expressed in \emph{Jerusalem}:

\poembox[2.25em]{%
	\enquote{I must create a system, or be enslav'd by another man's. \\
	I will not reason \& compare: my business is to create}\supercite{kazin:portable-blake}
}

Following this interpretation, the \enquote{poor indigent} is one who will not \enquote{reason \& compare}; these actions imply that
the \enquote{system} is being held in an inactive state; his creativity is simply a worldly
corollary to his \enquote{mystic} delight:

\poembox[2.25em]{%
	\dots the poor indigent is like the diamond which, tho' cloth'd \\
	In rugged covering in the mine, is open all within              \\
	And in his hallow'd center holds the Heavens of bright eternity.\supercite{keynes:william-blake}
}

The phrase \enquote{enraged beach} which ended the earlier quotation from Milton
is given meaning by the following sentence in that same \enquote{verse}:

\poembox[2.25em]{%
	Ozoth here builds walls of rocks against the surging sea,   \\
	and timbers crampt with iron cramps bar in the joys of life \\
	From fell destruction in the Spectrous cunning or rage.
}

\enquote{The surging sea,} and \enquote{the spectrous cunning or rage} are the same as \enquote{the enraged beach}:
although they represent something external to the man who is being considered, it is neither an
\enquote{opposing existence,} nor a wrongly perceived entity, since there was, to Blake, only
one existence, and therefore only one form (i.e., the true form) of perception;\footnote{See section II of this thesis.}
they represent what might be called \enquote{partially dead matter or entities,} which resist energy
by \enquote{absorbing} it, rather than by resisting it actively. It is necessary to understand
the metaphor and its implications, i.e., that the \enquote{enraged beach} represents the absence of
perception of reality, which absence is stated as \enquote{hardening} of the \enquote{optic nerve,} and
which cause has as one result the ceasing of creativeness, in favor of mere possession of a \enquote{system.}
To fully understand the earlier contention that the phrase \enquote{artificial riches} was not intended
to be read in a way that showed Blake to be allied with the supernaturalists, and to understand
the last quotation above, especially the words \enquote{\dots bar in the joys of life from fell destruction\dots,}
does not, in any way, imply that contact with the world is to be avoided. Rather that which is to be
avoided is that which would damage contact with (that is, perception of) the world.
Blake's most frequently used names for this \enquote{partial beath of matter} are \enquote{negation} and \enquote{spectre.}

Although this section has unavoidably extended into the subject of metaphysics, (as the discussion of metaphysics
will extend somewhat into the field of ethics) by its attempt to show the boundaries
of Blake's experiences and to contrast it to the experiences of the \enquote{supernaturalists,} its central
purpose has been simply to reveal that Blake's \enquote{experience} was one of the \enquote{total delight,}
which had its source in more ordinary delights, and that integral parts of the experience are
the feeling of expansion, unification, and opposition to error or death, while the experience may be
specifically defined as \enquote{seeing the world alive} and delighted and feeling one's self to be an integral part
of that infinitely joyful existence.

An \enquote{image} which seems to sum up this \enquote{experience} (which is itself a \enquote{summing up}) and indicates the
direction that will be taken by the following discussion of his metaphysics appears as the frontispiece of
\enquote{the gates of paradise}\supercite{keynes:william-blake}
and shows a caterpillar and a miniature infant in a cocoon on two leaves
of the same plant. That this is not intended to belittle man's value is revealed by the lines which
appear under it indicates that the character and value of all nature resides in the character of the perceiver,
so that a \enquote{mere} caterpillar can be considered as a creator of the \enquote{beheld} universe:

\bigskip

{
	\centering
	WHAT IS MAN?\par
}

\poembox[2.25em]{%
	The Sun's Light when he unfolds it \\
	Depends on the Organ that beholds it.
}

The \enquote{epistemological} significance that Blake gives to this \enquote{image} is shown by his placing
a reference to it alone below the first words of the title \enquote{the keys of the gates,} as follows:

\bigskip

{
	\centering
	THE KEYS\par
}

\poembox[2.25em]{%
	The Caterpillar on the leaf\\
	Reminds thee of thy Mother's grief.
}

\clearpage

\section{Blake's Metaphysics}

It has already been pointed out\footnote{See pp. \pageref{self:20}-\pageref{self:21} of this thesis.}
that Blake opposed supernatural religions and the mysticisms which accompany them,
but before proceeding with the central subject of this section, the \enquote{why} of Blake's experience which must include the
basic principles of existence, as well as psychology, as seen by Blake: it will be worthwhile to give
a general survey of Blake's statements which authenticate the use of the above definition of \enquote{metaphysics} rather
than the more popular conception of the word as meaning the study of that which is beyond the visible physical world.
The central part of this section will attempt to show that the combination of
\phantomsection%
\label{self:24}%
two factors necessarily result in
Blake's \enquote{ethics,} that is, in what has been called his \enquote{mysticism}: his passionate personal
involvement in the \enquote{perception and evaluation of truth}; following this, an attempt will be made to
show Blake's explanation for his, as well as anyone's, awareness of those two factors.

Of Blake's many references to Plato, most of which criticize his basic principles by showing his psychological
and ethical fallacies, two which appear in his marginal notes to Berkeley's \emph{Siris} are particularly
relevant to this discussion, by their rejection of \enquote{idealistic,} i.e., supernatural, philosophy. The first statement
is in response to Berkeley's statement:

\quotebox[2.1em]{%
	There is according to Plato properly no knowledge, but only opinion concerning
	things sensible and perishing, not because they are naturally abstruse and involved in
	darkness; but because their nature and existence is uncertain, ever fleeting and changing.\supercite{keynes:william-blake}
}

After one statement which has more to do with epistemology than with metaphystics, Blake says:

\poembox[1.9em]{%
	\dots that is the baseness of Plato \& the Greeks \& all warriors.
}

\noindent On another page of Berkeley, Blake made the note:

\quotebox[2.05em]{%
	\noindent What Jesus came to remove was the heathen or Platonic philosophy,
	which blinds the eye of imagination, the real man.\supercite{keynes:william-blake}
}

The preceding statements could be attributed to a misunderstanding of Plato's Philosophy, rather than to a rejection
of Plato's theory of ideas which exist more truly than the material world; however, the following cannot fail to be
recognized as a direct denial of supernatural \enquote{truth.} Referring to the passage by Swedenborg:

\quotebox{%
	\noindent\dots nothing doth in general so contradict man's natural favorite
	opinions as truth; and all the grandest and purest truths of Heaven must needs seem
	obscure and perplexing to the natural man at first view\dots\supercite{keynes:william-blake}
}

\noindent Blake says simply:

\poembox[2.2em]{%
	Lies \& priestcraft. Truth is nature.
}

Blake shows a reason for dualism in philosophy, when he says, \enquote{Thought without affection makes a distinction
between love \& wisdom, as it does between body \& spirit.}\supercite{keynes:william-blake}
Although the immediately preceding quotation seems to completely abolish the possibility of interpreting
Blake's philosophy as a dualistic or supernatural one. There can be cited statements
concerning \enquote{Hell} which suggest the source of confusion concerning his beliefs,
though they, in themselves, are clearly \enquote{monistic} and not superstitious. The one of these which
is more explicit follows:

\phantomsection
\label{self:03}

\quotebox[2.1em]{%
	\noindent To Hell till he behaves better. Mark that I do not believe there is such
	a thing literally, but Hell is the being shut up in the possession of corporeal
	desires which shortly weary the man. \emph{For all life is Holy}.\supercite{keynes:william-blake}
}

The meaning of this, it seems, would be more obvious if it were worded simply:
\enquote{I do believe such a thing literally, for all life is holy};
However, it does not seem to that it could be contradictory for Blake to believe in the sort of
Hell described in that central clause.\footnote{See pp. \pageref{self:22} of this thesis.}
It seems that the meaning of this is, \enquote{I do not believe
that it is Hell, in the usual negative sense, because all life is holy, but it is often called \enquote*{Hell}; rather,
this condition of being limited to corporeal desires is one that will make him behave better.} As mentioned in the
preceding section:\footnote{See pp. \pageref{self:23} of this thesis.}


\phantomsection
\label{self:31}

\poembox[2.25em]{%
	\dots Desire Gratified. \\
	Plants fruits of life \& beauty there.
}

In this case, being \enquote{shut up} in \enquote{the desires} would be a more forceful application of the principle. It is a
common characteristic of Blake to use a word in its ordinary, and frequently Platonic, sense, but with the opposite
emotional value; for instance, in the next example \enquote{angel} is in its
usual Christian-Platonic sense, allied with \enquote{reason,} but to Blake, \enquote{angels} and \enquote{reason} were \enquote{evil,} if that word can
be used. In this example, the \enquote{Hell} is neither called, nor intended to be, the same as that of the preceding example:
it is called \enquote{the infinite abyss,} and represents both the attitudes of \enquote{angels} and \enquote{reason} upon Blake, or
upon his \enquote{type.} In \enquote{a memorable fancy,}\supercite{keynes:william-blake}
the \enquote{angels} attitude is shown first:

\poembox[2.1em]{%
	An Angel came to me and said: \enquote{O pitiable foolish \\
	young man! O horrible, O dreadful state!}\supercite{keynes:william-blake}
}

\noindent After descending through a stable, a church, a mill, and a cave, Blake says:

\poembox[2.25em]{%
	\dots we beheld the infinite abyss, fiery as the smoke \\
	of a burning city\dots
}

\noindent When they have returned to the real world, Blake says:

\poembox[2.25em]{%
	\dots all that we saw was owing to your metaphysics: \\
	for when you ran away, I found myself on a bank      \\
	by moonlight hearing a harper.
}

Blake then shows the \enquote{angel} the corrupt nature of \enquote{Aristotle's analytics,} and, to the angel's complaint says:

\poembox[2.25em]{%
	We impose on one another, \& it is but lost time \\
	to converse with you whose works are only Analytics.
}

The last quotations show that the \enquote{Hell} which is not called such by Blake is only the
\enquote{non-existence,} the desire to transcend the Earth, of the supernaturalists; but through the
heaven of the supernaturalists is Blake's Hell, and vice versa, Blake is not a simple materialist. In Buddhist terms,
Blake \enquote{quit trying} to transcend the world and \enquote{succeeded,} through a more accurate term,
as revealed in the first section of this chapter, would be \enquote{transformed.}

On page \pageref{self:24} of this section, it was mentioned that the central part would be devoted
to showing how \enquote{the combination of two factors necessarily resulted in Blake's\dots} \enquote{mysticism,} i.e., his
pursuit of a superior form of consciousness. As Collin Wilson indicated, and Blake implied by
his poem \enquote{To Butts,} Blake could be considered a \enquote{vitalist}; that is, he believed that \enquote{life} was a universal
thing (\enquote{an expansive} principle or force) and that human life and consciousness are merely
manifestations of the \enquote{creative essence.} Probably a more fitting name could be \enquote{pantheism}; at least
this name will aid in the elucidation of that part of Blake's metaphysics which can be said to deal with,
without deviating much from the original meaning of that word, the \enquote{entelechy} of all matter, i.e., that
which may be called the \emph{elan vital}, or \enquote{cause,} in other words, that which makes actual,
or realizes, the merely potential, as distinguished from the part which will follow (mentioned in the section on
epistemology), viz., ontology, which is by its nature, very closely related to epistemology.

\phantomsection
\label{self:25}

\subsection{Entelechy}

A common, though entirely erroneous, understanding of \enquote{pantheism} is that it is synonymous with \enquote{animism.} Blake's
attitude toward animism was the same as his attitude toward any dualism; in \emph{The Marriage of Heaven and Hell}, he says
that after the \enquote{\dots ancient poets animated all sensible objects with Gods to geniuses\dots} \enquote{\dots some\dots enslav'd the vulgar
by attempting to realize or abstract the mental deities from their objects: thus began priesthood\dots}\supercite{keynes:william-blake}
Included in this sort of misuse of the word \enquote{pantheism} is that \enquote{philosophy} which might be called \enquote{higher animism,} which believes that all nature,
apparently excluding man (who has his own soul), has a \enquote{soul}: the only similarity this has to true pantheism is that it
tends to put nature on a more man-like level. Blake, of course, would reject this attitude because of its dualism, as he would
reject this attitude because of its dualism, as he would that slightly different form of the same belief which, although
it is something considered pantheistic because it does not consider the world to be evil, looks though, or beyond,
the world, to a world of Platonic ideas, etc., and the same \enquote{world-soul.}

A second, and somewhat more justifiable, criticism of Blake is that which claims that the \enquote{vitalistic} position diminishes
the value of God, either by making him impersonal, or by associating him with \enquote{our level} of things. Blake would
disagree with that form of pantheism which would require that \enquote{God} be worshipped as an abstract and universally equal principle
rather than as what might be called the expression of the vitalistic principle in particular
things; wherever abstraction exists it is hated by Blake. His statement which was stimulated by a similar thought in
Swedenborg's \emph{Divine Love} reveals his position clearly:

\quotebox[2.1em]{%
	\noindent Think of a white cloud as being holy, you cannot love it, but
	think of a holy man within the cloud, love springs up in your thoughts, for to
	think of holiness distinct from man is impossible to the affections.
}

Another illustration of Blake's belief in a personal God is found in \emph{The Marriage of Heaven and Hell},\supercite{keynes:william-blake}
when he says:

\quotebox[2.1em]{%
	\noindent The worshop of God is, honouring His gifts in other men, each according to
	his genius, and loving the greatest men best. Those who envy or calumniate great men hate
	God, for there is no other God.
}

This preceding quotation seems clearly to be vulnerable to that criticism of pantheism as \enquote{degrading
God to our level}; however, it should be noted that the words \enquote{honouring his
gifts} suggests a reiteration of the idea that the \enquote{expression of the principle, rather than
the principle abstracted from its expression, is to be worshipped,} with the result that
something larger than man can be called \enquote{God.} This idea will be further elucidated in the discussion
of Blake's ontology. Although, as suggested, the statement about \enquote{no other God} was softened by the implication
of the existence of a universal principle which, although it is not to be worshipped as itself, can be called \enquote{God,} Blake's
true \enquote{answer} to this sort of criticism, i.e., that \enquote{God if lowered,} is given in \emph{There is no Natural Religion} (second series)\supercite{keynes:william-blake}
as the summary of a discussion of \enquote{the infinite,} as follows: \enquote{Therefore God becomes as we are, that we may be as he is.} The meaning of this
should be obvious: the \enquote{infinite} God, as the \enquote{universal entelechy,} i.e., the \emph{elan vital}, has continuously
as its effect mankind, which, under particular \enquote{natural} circumstances, can be aware of its infinite real nature. As the \emph{elan vital}
is creative, mankind, when aware of infinite nature is creative.

\phantomsection
\label{self:16}

Blake's use of the word \enquote{infinite} (and \enquote{eternity}) might seem to be contradictory to his \enquote{anti-Platonism,} but it is clear
that he does not intend them to imply by it personal immortality, such as, for instance, that argued
by Orthodox Christians; \enquote{infinity} refers to a state of \enquote{absolutely real} perception. Blake succinctly explains
this idea in \emph{Jerusalem}\supercite{kazin:portable-blake}
as follows:

\poembox[2.25em]{%
	As the Pilgrim passes while the Country permanent remains. \\
	So men pass on: but States remain permanent for ever.
}

Since a simple and universal \enquote{vitalistic} impulse is the source of man's creativeness,
the universe conceived by Blake was not the \enquote{disorderly} one that it might be supposed when it is considered that each \enquote{identity} is a \enquote{God}: to
Blake's non-Platonic and thoroughly monistic mind, the only \enquote{general absolute} was that which here has been called the \enquote{entelechy}; though each \enquote{identity}
is completely free, it, and its creative acts, are expressions of the \enquote{entelechy} of
\enquote{No omnipotence can act against order.}\supercite{keynes:william-blake}
Nevertheless, \enquote{Each thing is its own cause and its own effect.}\supercite{keynes:william-blake}

The preceding discussion of the nature of \enquote{God} and \enquote{man} has, to some extent, answered the third objection that
Christians frequently raise against pantheism and against Blake, viz., that \enquote{personality,} or separate immortal identity
is lost in, or neglected by pantheism. It has been shown in contradiction to this criticism, that each identity is
considered to be a God, and is, in a way, \enquote{infinite.} It has also been shown that, although the \enquote{God} cannot truly be
said to be inferior to other sorts of gods, since the \enquote{states,} therein the infinity is expressed, \enquote{remain permanent forever,}
the criticism is valid in its saying that personal immortality is lacking from pantheism. However, Blake treats the subject
explicitly in a note on one of Swedenborg's statements:\supercite{keynes:william-blake}

\phantomsection
\label{self:14}

\quotebox[2.1em]{%
	\dots Essence is not identity, but from essence proceeds identity \& from
	one essence may proceed many identities, as from one affection may proceed many thoughts.
	Surely this (Swedenborg's statement that they are the same) is an oversight.
}

\quotebox[2.1em]{%
	That there is but one omnipotent, uncreate \& God I agree, but
	that there is but one infinite I do not; for if all but God is not infinite, they shall
	come to an end, which God forbid.
}

\quotebox[2.1em]{%
	If the essence was the same as the identity, there could be but one identity, which
	is false. Heaven would upon this plane be but a clock; but one \& the same
	essence is therefore essence \& not identity.
}

Thus, although Blake's conception of \enquote{identity} might be considered to be unsatisfactory
by an Orthodox Christian, who desires personal immortality after the death of his body, it seems to be, for
the person who desires \enquote{mystical} experience while \enquote{on Earth,} a superior position, since the \enquote{identity} itself
is \enquote{infinite,} i.e., unbounded and free: to the extent that it exists, and can perceive its goal, it is
capable of possessing it, or does possess it, while the
Christian conception of the soul which experiences \enquote{mystical union} necessarily involves a \enquote{journey} of some sort, to
extricate itself from its contamination with matter (see chapter I of this thesis).

\subsection{Ontology}

Having mentioned\footnote{See chapter III, section III, pp. \pageref{self:25} of this thesis.}
that ontology is very closely related to epistemology is, fundamentally, concerned with the idea that
truth is given immediately by means of the senses (including those which can be considered the \enquote{internal sense,}) and the best truth
by \enquote{inspired sense,} it should be obvious that Blake's ontology can be summarized by the statement \enquote{perception is
equivalent to existence.} In other words, \enquote{perceptualism} is that complement of \enquote{vitalism,} which turns what
would otherwise be simple pantheism into \enquote{mysticism,} or \enquote{mystical pantheism.} Conversely, \enquote{vitalism} is that force
(considered as mere \enquote{expansion,} in chapter III, section I., p. \pageref{self:26}: \enquote{my eyes did expand}) which \enquote{inspires} (or vitalizes) perception.
This interaction can be better understood by considering the psychological effects of holding either idea independently.
If \enquote{vitalism} were the only doctrine held, the person would supposedly be interested in the aggrandizement of living things, of
society, and o himself; there would be no necessary reference to consciousness. On the other hand, if \enquote{perceptualism} were the only intellectual predisposition, there would not necessarily be either a philosophical recognition of a creative impulse, physical
or psychological, or a practical interest in the world; in fact, this position is frequently coupled with pessimism, although in
perceptualism itself there is no reason for such a position. It is only when the recognition of an \enquote{expansive} principle is
coupled with perceptualism that a \enquote{worldly mystical} position results.

The \enquote{interaction} and its relation to the \enquote{mystical experience} can be understood with some perspective if the two factors
are considered in the situation referred to by Blake in the statement: \enquote{Each thing is its own cause \& its own effect,}\supercite{keynes:william-blake}
i.e., \enquote{within} the body (or any other entity). If the
\enquote{entelechy} is considered to be some sort of an \enquote{expansion,} as mentioned above, it is seen that each unit of the body will be exerting force continuously on the surrounding
units; the resulting contact is what Blake calls \enquote{perception.} Equating perception with existence (as is done by perceptualism), it is seen that a thing
\enquote{causes} itself of perceiving itself. Using Buddhist terminology, \enquote{expansion}\footnote{See chapter I, pp. \pageref{self:27} of this thesis.}
could be called \enquote{the primary reflex of the void,} and perception (or existence), \enquote{the greatest delight,} (Maha Guna).\footnote{Blakes \enquote{yoga} also was very similar to that of Tantric Hinduism and the related Buddhism; the main difference between the two seems to be that the Tantrists limited their worldly yoga or ethic to a sort of esoteric ritual, while Blake made the entire world, as far as it was known to him, the field for practice of his \enquote{mystical} yoga or ethic.}

In the situation of \enquote{worldly mysticism,} one's worldly actions will be directed toward the improvement of consciousness, on all levels, and
one's consciousness, i.e., one's existence, will be (according to Blake's beliefs) increased until, barring an increase in outside opposition, \enquote{\dots Man's
possession is infinite \& himself infinite.}\supercite{keynes:william-blake}
Since it seems obvious that the entelechy must be, in itself, unchangeable
in intensity, an explanation must be given for a weakening of its effect, and the resultant weakening of perception. Although
many things contribute to a lack of awareness of the factors of vitalism and perception\footnote{See pp. \pageref{self:28}-\pageref{self:22} of this thesis.}
the apparent weakening of the \enquote{expansive force} is called
by Blake simply lack of \enquote{organization}: (\enquote{innocence} will be treated more fully
in the third section of this chapter, but it, in general, means \enquote{awareness of and possession of the visionary faculty.})
This apparent weakening of the \enquote{entelechy} is an effective weakening of the desires: \enquote{\dots being restrain'd, it by degrees becomes passive, till it is only the shadow of desire.}\supercite{keynes:william-blake}
Although this subject will be quite thoroughly developed in the third section of this chapter, a distinction already made between the lack
of awareness of the entelechy's presence and the apparent lack of strength of the entelechy
can be illustrated: as already mentioned, the \enquote{lack of strength} is the result of some kind of \enquote{disorganization}; the \enquote{lack of awareness} is
caused by the possession of something which appears to be truth, but is either false or empty: speaking of that desire which has already been \enquote{restrained,} Blake
says \enquote{\dots the restrainer or reason usurps its place \& governs the unwilling.}\supercite{keynes:william-blake}
According to this, reason (\enquote{the same dull round over again}) becomes the limit of consciousness
(when desire is forgotten), and creative desire is forgotten and neglected.

\phantomsection
\label{self:28}

The quotation above (\enquote{\dots man's possession\dots}) will also help somewhat to show, as suggested at the beginning of this section, Blake's
explanation for human awareness (or discovery) of those two factors, viz., vitalism and perceptualism. The complete seventh proposition is \enquote{There is no Natural Religion,} second
series,\supercite{keynes:william-blake}
reads \enquote{the desire of man being infinite, the possession (the equivalent of \enquote*{perception}) is infinite \& himself infinite.}\supercite{keynes:william-blake}
(At this point, it is interesting to note the similarities among the philosophies
Blake, Buddhism,\footnote{See pp. \pageref{self:01}-\pageref{self:18} of this thesis.} and Maslow.\footnote{See pp. \pageref{self:29}-\pageref{self:30} of this thesis.}) Although this is primarily a statement concerning \enquote{essence} or entelechy, it is important to note
that the \enquote{expansion} is present in the body as desire, and, eventually, is recognized as equivalent to the entelechy, or \enquote{God}; a more specific
statement of the body's being the initiator of \enquote{mystical} consciousness is found in \emph{The Everlasting Gospel}, where \enquote{Jesus} calls the body:

\poembox[2.25em]{%
	Love's temple that God dwelleth in, \\
	\dots\dots\dots                     \\
	The naked Human Form Divine,        \\
	\dots\dots\dots                     \\
	On which the Soul\footnotemark expands its wing.\supercite{keynes:william-blake}
}

\footnotetext{\enquote{Soul} is used here to mean simply \enquote{consciousness}; Blake says: 1. Man had no body distinct from his soul\dots; 2. Energy is the only life, and is from the body\dots; 3. Energy is eternal delight.\supercite{keynes:william-blake}}

In other words, simply a degree of freedom to follow the desires of the body\footnote{See chapter III, pp. \pageref{self:31} of this thesis, and the third section of this chapter.}
constitutes the beginning of the \enquote{mystic way.}
Blake suggests that dualism is the barrier to the realization that the reality of all things depends upon the perceiver, when he says:

\clearpage

\phantomsection
\label{self:22}

\quotebox[2.25em]{%
	The ancient tradition that the world will be consumed in fire at the end of the six thousand years is the true,
	as I have heard from Hell. For the cherub with his flaming sword is hereby commanded to leave his guard at thee (sic.) of life;
	and when he does, the whole creation will be consumed and appear infinite and holy, whereas it now appears
	finite \& corrupt. This will come to pass by an improvement of sensual enjoyment.
	But first the notion that man has a body distinct from his soul is to be expunged\dots\supercite{keynes:william-blake}
}

This last line refers to the idea that the fact of perception is subsequent, or rather, subordinate, to the
entelechy, which is its cause, i.e., its \enquote{soul}; it is the failure to be conscious of the inter-relatedness of the two factors, end even
of their existence, which prevents one from making use of this knowledge to \enquote{cleanse} his perceptions,\footnote{\enquote{If the doors of perception were cleansed every thing would appear t oman as it is, infinite.}\supercite{keynes:william-blake}}
at least to the extent that he can while living in a society which still perceives nature as \enquote{finite \& corrupt} (\enquote{\dots some
scarce see nature at all.}\supercite{keynes:william-blake}) On the basis of the above quotation, Blake's ethics, which concern the social aspects of
\enquote{cleansing the perceptions} mentioned above could be called \enquote{eschatology,} i.e., a consideration of the destinies, especially in the
mystical sense, of man, as an individual and as a group. Ethics, in this sense, can be considered to be somewhat similar to karma yoga.

\section[Blake's Ethics, or Worldly Eschatology, Including the\\Cause, Character, and Effect of His Mysticism]{Blake's Ethics, or Worldly Eschatology, Including the Cause, Character, and Effect of His Mysticism}

As the \enquote{subjective} aspect of the \enquote{purification} of man's vision was the subject of the first section of this chapter, which showed that
Blake's form of \enquote{mysticism} was a permanent identification of consciousness (rather than a temporary escape from the world), the  \enquote{objective} aspect of
the \enquote{purification} of man's vision is the subject of this section; thus, the topic considered here could be called \enquote{the purification of society's vision,} \enquote{the
purification of action,} or, using the Blakean meaning of the words, \enquote{social mysticism.} However, as mentioned in the first section of the chapter, Blake considered the entire world,
as well as every particle of matter, to be \enquote{human,} so that he made no real distinction between the so-called \enquote{subjective} and the \enquote{objective.}\footnote{This \enquote{mystical} attitude towards society is the basis for some of the psychological analyses which consider Blake to be scizophrenic; ohwever, Jung's classification of this personality type as \enquote{introverted,} sane, and even superior to other types, is the one which is assumed, in this thesis, to be true.}
To understand Blake's \enquote{mysticism,} the whole must be seen at once--man must be viewed as an integral part of an absolutely real universe, wherein \enquote{error} or the apparent dislocation of any part of that
universe is constantly being rejected by the \enquote{inspired} parts of that universe, which are devoted to a perception of truth.

From the preceding discussions of Blake's ideas, it should be evident that Blake's \enquote{ethics} are largely concerned with political,
economic, and social goals, with, mainly, only negative attention to \enquote{morality.} The ideas of innocence and experience are the \enquote{keys}
to this section more than others, and, consequently, the \enquote{Songs} of Innocence and of Experience which elucidate these ideas will be given attention. Simply, primary
innocence and its corrupted state of \enquote{simple-minded innocence} is considered in \enquote{songs of innocence,} and an intelligent, realistic grasp of \enquote{the world
as it is} seems to be the basis for \enquote{songs of innocence}; complete, or \enquote{organized,} innocence is implicit in experience,
since when error and tyranny are recognized they will be destroyed, and in the freedom and abundance which will replace them pure innocence
(confidence) will be in line with the facts. Although capitalism is the main \enquote{political} subject criticized by Blake,
the religions which are allied with capitalism because of their authoritarian nature were equally hated by him.

\enquote{The Chimney Sweeper} in the \enquote{Songs of Innocence} is a particularly harsh condemnation of the society and economic system
which corrupts the original innocence of children who group up in it. The first stanza is direct statement of the power of money
over weak individuals (and my father sold me), and the second gives the exploiters argument for obedience and practicality, \enquote{hush, Tom, never mind it,} which
is the well-known lie that poverty is a happier state than wealth, because it avoids the problems of wealth. The first two lines
of stanza three suggest the mental result of an external effect; since there is no hope for the weak in the real world despite the rich-man's argument
that there is, church's offer of happiness in \enquote{the other world} is accepted.

Stanzas four and five are devoted to the church's happy myth of salvation which Tom saw in his dream, in the last stanza:

\poembox[2.25em]{%
	And so Tom awoke, and we rose in the dark,        \\
	And got with our bags \& our brushes to work.     \\
	Tho' the morning was cold, Tom was happy \& warm: \\
	So, if all do their duty, they need not fear harm.
}

Evil has been accepted and adapted to, so that it has become invisible. The last line shows, in Blake's opinion, how
religion supports the exploitation of the laboring class.

Three poems in the \enquote{songs of innocence} contain statements of a problem that is not revealed as a problem until
it is restated in the \enquote{songs of experience.} \enquote{Pity} is the theme of the three poems \enquote{The Divine Image,} \enquote{Holy Thursday,} and
\enquote{On Another's Sorrow}; although the idea of misery is inherent in that of pity, it is not emphasized here. It seems to be assumed that
distress is necessary, as, Blake believed, the church and the state want it to be assumed; for instance, \enquote{The Divine Image} begins:

\poembox{%
	To Mercy, Pity, Peace, and Love \\
	All pray in their distress;     \\
	And to these virtues of delight \\
	Return their thankfulness.
}

\enquote{Holy Thursday} contains two statements which will appear as problems later, in the lines:

\poembox[2.25em]{%
	The hum of multitudes was there, but multitudes of lambs, \\
	Thousands of little boys \& girls raising their innocent hands.
}

The Christian metaphor of a \enquote{lamb} (which is to be pitied) is apparently used
to indicate that this is written from a priestly point of view,\footnote{Blake later said in \enquote{an answer to the parson,} who asks \enquote{why of the sheep do you not learn peace?}: \enquote{because I don't want you to shear my fleece.}}
and only the word \enquote{but} hints that this is not right. The
1. Blake later said in \enquote{An Answer to the Parson,} who asks: \enquote{Why of the sheep do you not learn peace?} \enquote{Because I don't want
you to shear my fleece.}

\noindent The priest's statement continues:

\poembox[2.25em]{%
	Beneath them sit the aged men, wise guardians of the poor. \\
	Then cherish pity, lest you drive an angel from your door.
}

Thus, the emphasis is again on being thankful for pity, rather than considering that misery and poverty are not
necessary: as with \enquote{lambs} in the earlier lines, \enquote{guardians of the poor} can have,
if seen from the \enquote{experienced} point of view, a negative meaning, i.e., \enquote{keepers of the poor,}
in the sense of \enquote{slave-holders} and \enquote{creators of poverty.} \enquote{Angel} is generally a symbol of tyranny.

Further expression of the theme of pity and the misery which is apparently necessary is given in \enquote{On Another's Sorrow,} the
feeble argument that the presence, and pity, of \enquote{thy maker} gives comfort is given in stanzas four to eight, and in
the last stanza the inefficacy of this is indicated, though still not explicitly state. The last two stanzas still illustrate how this is done:

\poembox[2.25em]{%
	Think not thou canst sigh a sigh, \\
	And thy Maker is not by:          \\
	Think not thou canst weep a tear, \\
	And thy Maker is not near.        \\
	~                                 \\
	O! He gives us His joy,           \\
	That our grief He may destroy:    \\
	Till our grief is fled \& gone    \\
	He doth sit by us and moan.
}

The \enquote{maker is near,} and by the \enquote{innocent} reader it is assumed that his efforts are the cause of its fleeing; however, this
is not what Blake said: he simply, as in the other poems of this book, allows it to be read in that way, since he is illustrating
that point of view. The interpretation that is supported by his other writings (Songs of Experience, Marriage of Heaven and Hell) is that
two are grieving where before only one was, that is, the problem had been doubled, rather than solved. The \enquote{Songs of Experience,}
besides refuting and offering solutions to problems in the \enquote{Songs of Innocence,} point out other evils, but here they are not given an
acceptable appearance; they are directly stated, and a solution more or less directly given. As mentioned earlier, the \enquote{Songs of Innocence} are
intended to illustrate \enquote{an intelligent realistic grasp of the world as it is}; for example, one poem, \enquote{The Angel,} gives an
outline of how the corruption of tyranny is to be thrown off, as \enquote{The Chimney Sweeper} in \enquote{Songs of Innocence} gave an outline of the
process of accepting tyranny. Others, such as \enquote{The Little Girl Lost} give glimpses of the world after it has, through revolution, throw off tyranny.

The \enquote{introduction} to this group of verses, and \enquote{Earth's Answer} give the basic idea of the book. In the first line of the \enquote{introduction} Blake
introduces himself, indicating that he is no longer speaking from a different viewpoint, or personality, and that he perceives reality
(\enquote{Who present, past, and future, sees}), even though his:

\poembox[2.25em]{%
	\dots ears have heard                \\
	The Holy Word                        \\
	That walk'd among the ancient trees; \\
	~                                    \\
	Calling the lapsed soul,             \\
	and deeping in the evening dew;
}
The \enquote{Holy Word} that \enquote{weeps} is apparently the one earlier called \enquote{thy maker.} The following line is somewhat confusing,
since, apparently to avoid \enquote{who,} he uses the word \enquote{that} to indicate the \enquote{Bard}:

\poembox[2.25em]{%
	That might control \\
	The starry pole,   \\
	And fallen, fallen light renew!
}

With this interpretation, \enquote{renewing the fallen light} would happen in spite of having heard \enquote{The Holy Word,} rather
than being done by \enquote{The Holy Word,} (not the \enquote{Bard}) which is \enquote{Calling the lapsed soul and weeping\dots} This interpretation
(rather than that which would say that Blake, the Bard, has heard the \enquote{poetic spirit} (Holy Word) which is calling the \enquote{lapsed soul,}
i.e., the unrealistic mind, back to reality, in which the \enquote{fallen light} would be renewed) which interpretation would
apparently be supported by the following lines:

\clearpage

\poembox[2.05em]{O Earth, O Earth, return!}

{
	\centering
	and\par
}

\poembox[2.15em]{Turn away no more}

Thus equating \enquote{Earth} with \enquote{lapsed soul,} seems necessary because in \enquote{Earth's Answer,} \enquote{Earth} speaks
with extreme harshness of \enquote{\dots the father of the ancient men} who is \enquote{weeping,} for example:

\poembox[2.15em]{%
  Selfish father of men \\
  Cruel, jealous, selfish fear
}

{
	\centering
	and\par
}

\poembox[1.9em]{%
  Selfish! vain! \\
  Eternal bane!
}

If this were not intended to refer to \enquote{thy maker,} \enquote{the Holy Word,} as opposed to the Bard, it could
only mean that \enquote{Earth} was rejecting not only the \enquote{Holy Word,} but also the renewal of the \enquote{fallen light}
and the \enquote{return} mentioned, while the main purpose of the poem is obviously to ask for these things, which
signify \enquote{liberty}:

\poembox[2.1em]{%
	Earth rais'd up her head,                \\
	From the darkness dread \& drear.        \\
	Her light fled:                          \\
	Stony dread!                             \\
	And her locks cover'd with grey despair. \\
	~                                        \\
	Can delight                              \\
	Chain'd in the night                     \\
	The virgins of youth and morning bear.   \\
	~                                        \\
	Does the sower?                          \\
	Sow by night?                            \\
	Or the plowman in darkness plow?
}

\clearpage

\noindent thus it is seen that the lines:

\poembox[2.1em]{%
	O Earth, O Earth, return!     \\
	Arise from out the dewy grass! \\
	Night is worn,                 \\
	And the morn                   \\
	Rises from the slumbrous mass. \\
	~                              \\
	\enquote{Turn away no more;    \\
	Why wilt thou turn away?}
}

\noindent are spoken by the Bard himself, and as himself, not as the public voice
of the \enquote{Holy Word,} and are asking the world to turn away from the Holy Word's
control and to turn away no more from reality. The \enquote{lapsed soul} could, without departing
from Blake's principle (chapter III, metaphysics) that the soul and body are not separate
things, be interpreted as the \enquote{Earth} which is being called into unreality, or corruption, from
its original innocent state by the dualistic philosophy of tyranny. It will be noted that this
differs from the equation of \enquote{Earth} with \enquote{lapsed soul} in the rejected interpretation in the
direction that the world is being \enquote{called} by the Holy Word, which in the rejected interpretation
was \emph{into reality} thereby giving the Holy Word a positive function, which, as was explained,
would make the poem \enquote{Earth's Answer} meaningless and self-contradictory and would also be contrary
to Blake's total philosophy.

\enquote{The Clod and the Pebble} is one of the most concise, yet complete statements of the difference between the
so-called \enquote{innocent} state and the \enquote{experienced} state, and the cause of each. The first stanza:

\poembox[2.25em]{%
	Love seeketh not itself to please, \\
	Nor for itself hath any care,      \\
	But for another gives its ease,    \\
	And builds a Heaven in Hell's despair.
}

\noindent is typical of the poems in the \enquote{Songs of Innocence,} but two key words give an exact
and unavoidable meaning to it even without the rest of the poem, although those stanzas add to the clarity
of its meaning. An \enquote{innocent,} or \enquote{angelic,} reading for the last line would, of course, be \enquote{and build a Heaven in spite
of Hell's opposition,} or, more in agreement with the poem, \enquote{and builds a Heaven within Hell, or upon a desperate situation,}
viz., being \enquote{trodden with the cattle's feet.} However, that this is not the meaning of the words \enquote{Hell's despair} is
indicated, if the fact that being trodden by cattle's feet is a Blakean description of Hell is insufficient
indication by the ambiguous word applied to Heaven in the last stanza, i.e., \enquote{despite,} meaning, approximately,
\enquote{malice.} If there is to be any real distinction between Heaven and Hell, it is obvious that Hell can not be
a place of despair in the above sense and Heaven a malicious force; therefore, the only other possible interpretation
of \enquote{Hell's despair} must be used, and this is the meaning that should be obvious to a reader familiar with Blake. Since,
for instance, Blake says in \enquote{The Marriage of Heaven and Hell,}\supercite{kazin:portable-blake}
\enquote{Good is the passive that obeys reason. Evil is
the active springing from energy,} and \enquote{Energy is eternal delight,} \enquote{Energy is the only life\dots,} and \enquote{Good is Heaven. Evil is Hell.}
Thus his Hell is \enquote{energy} and \enquote{eternal delight.} Obviously, the phrase \enquote{Hell's despair} is intended to mean
that after having seen \enquote{trodden with the cattle's feet,} that is, tyrannized and made miserable, \enquote{energy} may lose hope, and allow
the individual, whose passive qualities remain, to be controlled or tyrannized by whatever force
that seeks to impose a system of beliefs (\enquote{reason}) and action on mankind. As mentioned in connection
with the poem \enquote{On Another's Sorrow} in \emph{Songs of Innocence}, this type of \enquote{love} is not creative since
all it \enquote{builds} is a \enquote{Heaven,} which existed anyway, before Hell disappeared; rather, it is self-destructive
since it \enquote{gives its ease} for another. In the second half of the middle stanza:

\poembox[1.9em]{%
	But a pebble of the brook \\
	Warbled out these metres meet
}

\enquote{A Pebble of the Broom} refers to, apparently, anyone who is not subject to tyranny or continual troubles,
although the fact that a pebble is made of firmer material than a clod may not be simply accidental, that is, it
might be an intentional choice eased on Blake's opinion that \enquote{genius} is at least congenital if not
hereditary, rather than acquired. That this might have been his intention is suggested by his statement, \enquote{those who
restrain desire, do so because theirs is weak enough to be restrained, and the restrainer or reason usurps
its place and governs the unwilling. And being restrained, it by decrees becomes passive\dots}\supercite{kazin:portable-blake}
and by his underlining of Lavater's aphorism, \enquote{He alone has energy that cannot be deprived of it.}\supercite{keynes:william-blake}
Although the first example speaks of the person restraining his own desire, it is appropriate, since, as mentioned earlier, in relation to \enquote{The Chimney Sweeper,}
the beliefs and standards of the \enquote{oppressor} are accepted as those of the \enquote{oppressed}; however, this possible idea of the immunity
of genius from corruption by its environment is somewhat irrelevant to the theme of the poem, i.e., that oppression does occur,
and if it was intended as an additional theme, it does not seem to make the poem better, since the last stanza
is, at the most, only a partial statement of Blake's, or the genius's, actual beliefs. Another possible explanation for the
different \enquote{characters,} which would avoid the need of the last stanza's being an accurate statement of the \enquote{genius's attitude,} would be
that Blake intended only to attach a negative quality to the first condition, \enquote{reason} and corruption, rather than to show a specific
superiority of the second condition; his description of \enquote{reason} (Urizen) in chapter III of the First Book of Urizen, \enquote{Urizen is a
clod of clay,} uses the same metaphor. The stanza:

\poembox[2.25em]{%
	Love seeketh only Self to please, \\
	To bind another to its delight:   \\
	Joys in another's loss of ease,   \\
	And builds a Hell in Heaven's despite.
}

\noindent is, of course, the reverse of the first stanza, but it deserves some separate comment. It was said that
the type of love described in the first stanza was self-destructive and apparently non-creative. That this stanza
describes a type of love which is creative and not self-destructive is indicated by the second line, which suggests
that in attending to one's own pleasure another is involved in the delight by the fourth line, which says that
a Hell, a positive condition, is established in spite of \enquote{Heaven's despite,} or malice, malignity, and also
by the third line which particularly deals with this love's being not self-destructive, by showing that it does
not lose its joy, even when another has lost his. There would be no philosophical reason behind a reading of it as \enquote{gets
pleasure from another's misery.}

\enquote{Holy Thursday} is a refutation of the \enquote{priestly} thoughts contained in the poem of the
same name in \emph{Songs of Innocence}, although the idea given in the last line, \enquote{Then cherish pity\dots,} etc., is omitted from consideration,
probably because it was involved in \enquote{The Clod and the Pebble,} which immediately precedes this poem, and also is thoroughly
covered later in \enquote{The Human Abstract.} The line \enquote{now like a nightly wind they raise to Heaven the voice of song,} of the \enquote{innocent}
version is answered by the lines:

\poembox[2.25em]{%
	Is that trembling cry a song? \\
	Can it be a song of joy?      \\
}
	
\noindent and the lines which follow: 

\poembox[2.25em]{
	And so many children poor? \\
	It is a land of poverty!
}

Apparently refer to the phrase in the earlier poem, \enquote{\dots guardians of the poor},
with the interpretation given earlier. That this was the meaning intended is indicated by the
first stanza:

\poembox{%
	Is this a Holy thing to see, \\
	In a rich and fruitful land, \\
	Babes reduc'd to misery,     \\
	Fed with cold and usurous hand?
}

Which, although it does not directly implicate a statement in the other poem, as does the second stanza, it
seems likely that the \enquote{\dots cold and usurous hand} refers to that of the beadle, holding
the mace of church authority, the \enquote{wand} \enquote{\dots as white as snow\dots} Since \enquote{authority} is opposed to human
energy, warmth, and freedom, \enquote{snow} is used as a description for the wand, symbol of authority and oppression.
The association of oppression with cold and winter is probably the reason for using the word \enquote{trembling} in the
second stanza, and must be the \enquote{key} to the third and fourth stanzas:

\poembox[2.25em]{%
	And their sun does never shine.        \\
	And their fields are bleak \& bare.    \\
	And their ways are fill'd with thorns. \\
	It is eternal winter there.            \\
	~                                      \\
	For where-e'er the sun does shine,     \\
	And where-e'er the rain does fall:     \\
	Babe can never hunger there,           \\
	Nor poverty the mind appall.
}

The third stanza describes the existing conditions, but the last seems to be loosely description drawn, in fact
a minimum, of his ideal society (which he later calls Jerusalem). The sun in these stanzas, it seems obvious, represents
\enquote{human energy and desires}; \enquote{rain} possibly stands for the \enquote{grateful tears} mentioned in the poem in the \enquote{second series};
\enquote{Ross. Pick. mss.}

\poembox[2.25em]{%
	The Sun is freed from fears, \\
	And with soft grateful tears \\
	Ascends the sky.
}

The implication is, regardless of the details, that when oppression
is removed (destroyed) poverty will disappear.

Although there is probably some support for the opinion that Blake became discouraged with the
possibility of violent revolution as a means of establishing his \enquote{Jerusalem} in England, and supposedly
in the world, as late as the writing of \emph{Milton}, 1804-1808, revolution is a prominent theme, and his
last major work, \emph{Jerusalem}, definitely contains implications of the necessity for revolution,
although the emphasis seems to be largely on the psychological gains of the revolution. Practically all of
the \enquote{didactic and symbolic works} propose revolution of some sort, i.e., they attempt to inspire a desire
for freedom from any sort of oppression, including that of reason, and at least nine of them are largely
calls for political revolution. \enquote{The Little Girl Lost} and \enquote{The Little Girl Found,} in \enquote{Songs of Experience,} seem
to be early examples of this characteristic of Blake's thought. Although much of each poem is devoted to an analysis
of the social prerequisites for revolution, the basic idea, that the world will awaken, discover \enquote{vision,} and
through it see the revolution, with the result that the world will become a beautiful (rather than harsh) place,
is clearly revealed in certain stanzas. In developing this idea Blake adds to and alters what earlier
appeared as two ideas, \enquote{pity} and \enquote{thy maker,} and in doing so reverses the meaning of those words previously
used for \enquote{angelic} ideas, and binds them into a single concept, one which later became \enquote{Jesus,} or the
\enquote{eternal humanity,} and is the perfect, creative rebel.

In the opening two stanzas of \enquote{The Little Girl Lost,} which two stanzas apparently serve as an
introduction to the two poems, and seem to build on the idea of the closing stanza of the preceding poem,
\enquote{Holy Thursday,} Blake says:

\poembox[2.25em]{%
	In futurity                \\
	I prophetic see.           \\
	That the Earth from sleep. \\
	(Grave the sentence deep)  \\
	~                          \\
	Shall arise and seek       \\
	For her maker meek:        \\
	And the desart wild        \\
	Become a garden mild.
}

Subsequent stanzas will reveal that \enquote{her maker meek} refers not to the \enquote{Holy Word} or a similar
entity, but to the individual whose energetic faculties are in control, and since the word \enquote{meek} is
applied to a creative thing, i.e., as opposed to the grave (sentence of sleep) the effect is to temper,
or rather, to show to be safe the thing it refers to rather than to indicate the only quality of an ineffective
character, as the pity of \enquote{thy maker} in \enquote{On Another's Sorrow,} the sixth stanza, with the lines:

\poembox[2.25em]{%
	Lost in desart wild \\
	Is your little child.
}

\noindent describes a child's condition in England, or in any other country, which is
economically, philosopically, and psychologically a \enquote{desart}; the last of this stanza and all
of stanza given continue showing the inefficacy of the kind of pity revealed in \emph{Songs of Innocence};
although the parents \enquote{weep,} it is the \enquote{beasts of prey} that first \enquote{view'd the maid asleep,} stanza nine.

The character, benevolent yet fiery, playful in youth, intense and gently loving in old age, of the \enquote{mystic
(or visionary) rebel} is portrayed in the eleventh and twelfth stanzas:

\poembox[2.25em]{%
	Leopards, tygers play   \\
	Round her as she lay;   \\
	While the lion old,     \\
	Bow'd his mane of gold, \\
	~                       \\
	And her bosom lick,     \\
	And upon her neck,      \\
	From his eyes of flame, \\
	Ruby tears there came;
}

The following, and last, verse of his poem indicates that the \enquote{beasts of prey} have, with their
condition of vitality and compassion, a practical nature:

\poembox[2.25em]{%
	\dots the lioness         \\
	Loos'd her slender dress, \\
	And naked they convey'd   \\
	To caves the sleeping maid.
}

A clear contrast is obtained here, by having the first line of the next poem reiterate the idea of the inefficacy of the
no-visionary kind of pity, which, by itself (in the personality lacking use of the visionary faculty), does nothing but add to the total woe:

\poembox[2.25em]{%
	All the night in woe \\
	Lyca's parents go    \\
	Over vallies deep,   \\
	While the desarts weep.
}

This is the same sort of event mentioned in the poem already discussed, \enquote{On Another's Sorrow,} e.g., \enquote{He becomes
a man of woe\dots} By the fifth stanza the woman's \enquote{weary woe} has become overwhelming: \enquote{She could no farther go.}
The next lines evidently represent the beginning of progress toward the visionary state, which will be tremendously different from
that state in which, in the third and fourth stanzas, they \enquote{dream} (\enquote{among shadows deep}) and see a \enquote{pale} \enquote{fancied image,} since
the man exhibits his interest for another in a practical way, and is then confronted by a lion, i.e., a \enquote{visionary,} or \enquote{vision} itself:

\poembox[2.25em]{%
	In his arms he bore          \\
	Her, arm'd with sorrow sore; \\
	Till before their way        \\
	A couching lion lay.
}

Although the \enquote{visionary experience} may be fearsome, it will lead the person on, deeper into its \enquote{control.} The narrative
continues with this idea:

\poembox[2.25em]{%
	Turning back was vain, \\
	Soon his heavy mane,   \\
	Bore them to the ground.
}

Their fearfulness of the experience is dispelled by the experience itself:

\poembox[2.25em]{%
	\dots their fears allay    \\
	When he licks their hands, \\
	And silent by them stands.
}

Each step, and each stanza, leads them to a higher level of confidence in understanding the visionary faculty (the lion), and in
succeeding stanzas they \enquote{\dots behold a spirit arm'd in gold,} lose \enquote{\dots all their care,} follow \enquote{\dots where the vision
led} (and, doing so, apparently sop weeping \enquote{for the maid}), to his palace, where the child had been taken earlier, although then,
it had appeared to them as a cave. The last stanza might depict the \enquote{garden mild} prophesied in the \enquote{introduction} to these poems, although,
since it says that:

\poembox[2.25em]{%
	To this day they dwell \\
	In a lonely dell
}

\noindent it probably means that they have become visionary rebels, members of a small group which is helping to \enquote{lead} the rest
of society out of its \enquote{desart,} as the \enquote{lions} led them. It is more explicitly shown in one of Blake's \enquote{Proverbs of Hell}:

\quotebox[2.25em]{%
	The roaring of lions, the howling of wolves, the raging of the stormy sea, and
	the destructive sword, are portions of eternity too great for the eye of man.\supercite{keynes:william-blake}
}

\noindent that he considered the \enquote{lion} to be representative of a superior state. This superior state, as shown in the above
quotation, is meaningless, or non-existent (or fearsome, if its characteristics are seen) to an ordinary person, but its characteristics,
mentioned in the last lines of the poem as \enquote{\dots the wolvish howl\dots the lions growl,} are not feared by those who have come to live in the
\enquote{dell,} that is, those who are also visionaries. The meaning that Blake is expressing in the poem and proverbs quoted above is the same
as that contained in the well-known Hindu story of the \enquote{roar of the tiger,} viz., that the tiger, which was raised believing himself to be a goat,
eating grass and bleating, happened to discover the taste of flesh whereupon he \enquote{uttered the tiger's roar of self-realization,}
realizing himself to be part of \enquote{eternity,} or \enquote{the creative void.}

\phantomsection
\label{self:02}

\noindent Another of Blake's \enquote{Proverbs of Hell}:

\poembox[2.25em]{%
	The tygers of wrath are wiser than the horses of instruction.
}

\noindent and, less explicitly, the poem \enquote{The Tyger,} in \enquote{Songs of Experience} (especially the description \enquote{\dots burning bright
in the forests of the night\dots}) are further examples of Blake's practice of using a tiger or lion to represent \enquote{visionary}
or \enquote{mystical} inspiration, as opposed to the sheep, lamb, or horse, which represent corrupted, controlled, weakness and the rational
knowledge and dualistic beliefs associated with it.

There are five other poems in the \emph{Songs of Experience} which are based on \enquote{political} themes: \enquote{The Chimney Sweeper,}
\enquote{The Little Vagabond,} \enquote{London,} \enquote{The Human Abstract,} and \enquote{A Little Bot Lost}; of these, only \enquote{A Little Boy Lost} begins with a quotation of \enquote{natural wisdom}
from the little boy; the others are only angry attacks on those things that lack \enquote{visionary energy} and sustain themselves on the lives and energies of others, viz.,
those whom they dominate. Doubt and reason (Urizen), God, the only virtuous (the accuser, called Nobodaddy, Satan, or Jehovah), the king, and priests, are Blake's favorite \enquote{mental enemies.}

Blake's \enquote{visionary mysticism,} it should be noted, denied dualistic philosophy and its basically pessimistic and skeptical attitude toward
the perceived world, but also denied a mechanistic materialism which is ultimately agnostic, thus being the most \enquote{gnostic} (and most truly \enquote{mystical}) of
the three positions. This visionary mysticism is apparent as the world view behind the poem \enquote{The Human Abstract} which specifically treats the politico-religious habit of piously
pretending to feel pity and mercy, while being really the cause of the misery and wars and enjoying the dual nature of society. The poem begins on this subject, directly and simply:

\poembox[2.25em]{%
	Pity would be no more             \\
	If we did not make somebody poor; \\
	And Mercy no more could be        \\
	If all were as happy as we.
}

However, the general pattern of the poem is one of an apparently causal sequence, from the fact of rationalized social dualism,
through exploitation, hypocrisy (of some sort), and humility, to \enquote{mystery,} thus explaining, to some extent,
the \enquote{mistaken} mentality (\enquote{\dots More! More! Is the cry of a mistaken soul. Less than All cannot satisfy Man.}) which accepts itself as being
less than complete. Following the line:

\poembox[2.2em]{%
		and mutual fear brings peace
}

\noindent which is the last part of the first stage of the sequence, is a line which is slightly ambiguous: \enquote{selfish loves} in:

\poembox[1.9em]{%
	Till the selfish loves increase
}

\noindent is seen to have a negative meaning (as do pity and mercy) when it is remembered that the four \enquote{virtues} listed in
\enquote{The Divine Image} are \enquote{Mercy, Pity, Peace, and Love.}\supercite{kazin:portable-blake}
Further indication of the negative meaning intended is the second half of that stanza:

\poembox[2.25em]{%
	Then Cruelty knits a snare, \\
	And spreads his baits with care.
}

That this is not the \enquote{cruelty} (or the \enquote{selfish love}) of active pride is indicated by its use of a snare, which is an
inactive method of \enquote{warfare}; also, that the cruelty is of the wealthy class upon the poor class is indicated by the line from
the Rossette-Bickering manuscript, \enquote{They cannot spread nets where a harvest yields.}\supercite{kazin:portable-blake}
The next verse shows that \enquote{Holy fears} (and tears) promote
the growth of \enquote{humility,} which appears in the fourth stanza as \enquote{\dots the dismal shade of mystery\dots} (The \enquote{tree} that grows \enquote{\dots in the human brain}).
That this is the same sort of mystery used by the churches is suggested by the last half of that verse:

\poembox[2.25em]{%
	And the caterpillar and fly \\
	Feed on the Mystery.
}

\noindent when the comparison of a \enquote{caterpillar} with a priest in the \enquote{proverbs of Hell}\supercite{kazin:portable-blake}
is remembered.

The simplification, or generalization, of the problem of spirituality based on political conflict, that which underlies
the ideas of innocence and experience, and which was the basis for \enquote{The Human Abstract,} is made more explicit in, is, in fact,
the entire subject of, the next to the last poem in \emph{Songs of Experience}, \enquote{A Little Boy Lost.} That subject is, of course,
the apparently necessary conflict between the forces of mystery and the forces of visionary knowledge.

The philosophy of uncorrupted innocence, the \enquote{anti-agnostic} argument of the \enquote{visionary mystic,} is spoken by the
little boy in the first two stanzas:

\poembox[2.25em]{%
	Nought loves another as itself,  \\
	Nor venerates another so,        \\
	Nor is it possible to Thought    \\
	A greater than itself to know:   \\
	~                                \\
	\enquote{And Father, how can I love you  \\
	Or any of my brothers more?      \\
	I love you like the little bird  \\
	That picks up crumbs around the door.}
}

This is the basic philosophy\supercite{kazin:portable-blake}
which was outlined in \enquote{There is no Natural Religion,} second series,
especially numbers V, VI, and VII,\supercite{kazin:portable-blake}
and in such statements as the proverb in \emph{The Marriage of Heaven and Hell},\supercite{kazin:portable-blake}
\enquote{One thought, fills immensity.}

After having seized the child, in the third stanza, the priest, in the fourth stanza, calling him a fiend,
says the child is:

\poembox[2.25em]{%
	One who sets reason up for judge \\
	Of our most holy Mystery.
}

In this case, \enquote{reason,} rather than being a \enquote{limiter,} or a \enquote{restrictor of knowledge,} is merely a controlled presentation
if the facts; although in this instance it could not be otherwise, since it is speaking of its own limits, the use of reason is
acceptable if it is remembered that, as Blake said in \emph{Milton}:

\poembox[2.1em]{%
	\dots The Reason is a State Created to be Annihilated, \& a new Ratio Created.\supercite{kazin:portable-blake}
}

In other words, reason must be maintained as a tool for describing reality, either of the imagination or of the \enquote{world,} and
prevented from becoming a system of empty formulae, whose only function is to control, to their detriment, the masses of people.
It is unnecessary to give an outside example of Blake's attitude toward \enquote{mystery,} the context of this poem being sufficient explanation,
but it is interesting to note some of the names given to it, such as, in \emph{Milton}:

\poembox[2.25em][2.15em]{%
	\dots Mystery of the Virgin Harlot, Mother of War\dots\supercite{kazin:portable-blake}
}

As was to be expected, since the visionary was a child, and thus apparently not completely \enquote{organized} in his
innocence, and this \enquote{corruptible,} the child is first \enquote{bound\dots in an iron chain} (stanza five) and then he is
burned (last stanza) \enquote{\dots in a holy place}:

\poembox[2.2em][2em]{%
	Where many had been burn'd before; \\
	The weeping parents wept in vain.  \\
	Are such things done on Albion's shore?
}

Thus, the forces of mystery have destroyed a part of life, because \enquote{innocence,} the \enquote{mystical} part of society,
is insufficiently organized; it is to remedy this situation that Blake said, \enquote{Rouze up, O young men of the new age. Set your foreheads against the ignorant hirelings: fore we have
hirelings in the camp, the court and the university, who would, if they could, for ever depress mental and prolong corporeal war.}\supercite{kazin:portable-blake}
\enquote{Unorganized mysticism,} Blake would say, referring to both society and the individual, \enquote{is an impossibility in a corrupt, non-mystical world.}

The first chapter of this thesis, in order to provide a meaning for the word \enquote{mysticism,} has traced the etymology
of that word to various roots whose meanings contain implications of intoxication, sensuality, and sensousness; the
history of mysticism is further traced through typical Oriental and Western expressions, and it is shown that removal
from the primitive conditions of mystical ritual has generally resulted in the application of the word (or its cognates and/or synonyms) to
opposite references, viz., world rejecting, sense rejecting, symbolic yet ineffable, \enquote{experiences,} or,
more accurately, \enquote{non-experiences.} And that, in both the orient and the occident, reaction and readjustment to the
resultant techniques for \enquote{ameliorating life} to society in general, rather than to a more or less
limited group. It is also shown that language and symbols are closely involved with the \enquote{negative mystical} practices,
and it is suggested that a positive attitude toward material production is closely related to, rather than opposed to,
mystical activity, if the \enquote{liberal} sense of that word is used.

The second chapter has considered and classified the writings on Blake's mysticism, and maintained that those
writers who have considered Blake's mysticism to be in the \enquote{world rejecting} tradition are mistaken,
and that the reason for this confusion is that, until nearly the middle of this century, only students of mysticism were aware of a precise
definition of that word, with the result that Blake's ideas were, in general, not closely studied. It was further mentioned that
merely indicating that Blake's mysticism is different from the \enquote{world rejecting} sort is of little value to an understanding of his ideas.

The third chapter, after illustrating that Blake was aware of a perceptual epistemology, stated that the source of Blake's
\enquote{liberal} or \enquote{visionary} mysticism is mainly in the combination of two philosophical commitments, viz.,
vitalism and perceptualism, which interact to produce an apparently wide variety of doctrines (which could be roughly
classified with the \enquote{actuality theory} of Heraclitus, Spinoza, Mach, and Whitehead), including Blake's \enquote{anthropomorphism,}
his \enquote{expanding sensation-expanding particles} idea, his rejection of both the Platonic and the Newtonian time theories, in favor
of an Einsteinian time theory, and the closely related doctrines of unbounded \enquote{identities,} and the creation
of each thing, momentarily, by itself. The last mentioned doctrine is more easily understood by using the concept of \enquote{entelechy,}
\enquote{that which makes actual.} Finally, it is maintained that, in Blake's  \enquote{visionary mysticism,} the parallel to an \enquote{eschatology} is
to be found in his \enquote{social mysticism,} that is, that instead of the \enquote{way} of self-abnegation, world rejection, and ultimately of
\enquote{residing with (or being married to) God,} his mysticism involved self-gratification, world interest, and, ultimately, a
reshaping of the world in line with the mystic's desires. It is by these features, as well as the visionary, \enquote{anthropomorphic,}
vitalistic and perceptual features, that Blake's mysticism is seen to be allied with the primitive, Oriental, and other \enquote{liberal mysticisms.}